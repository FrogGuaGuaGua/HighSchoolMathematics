\chapter{解三角形与向量}

\section{向量的数量积与向量积}
\begin{itemize}[leftmargin=\inteval{\myitemleftmargin}pt,itemsep=
   \inteval{\myitemitempsep}pt,topsep=\inteval{\myitemtopsep}pt]
\item 对于$ n $维向量$ \vec{a}=(a_1,a_2,\cdots,a_n) $,
$ \vec{b}=(b_1,b_2,\cdots,b_n) $,这两者的数量积
(也称为“标量积”,“内积”,“点乘”)定义为
\begin{gather*}
    \vec{a}\cdot \vec{b}=a_1b_1+a_2b_2+\cdots+a_nb_n
\end{gather*}

\item 对于三维向量$ \vec{a}=(a_1,a_2,a_3) $,
$ \vec{b}=(b_1,b_2,b_3) $,这两者的向量积(也称为“矢量积”,“外积”,“叉乘”)定义为
\begin{align}\label{三维向量叉乘}
    \vec{a}\times\vec{b} =
    \begin{vmatrix}
        \vec{i} & \vec{j} & \vec{k} \\
        a_1 & a_2 & a_3 \\
        b_1 & b_2 & b_3
    \end{vmatrix}
    =(a_2b_3-a_3b_2,a_3b_1-a_1b_3, a_1b_2-a_2b_1)
\end{align}
其中,$ \vec{i},\vec{j},\vec{k} $分别是$ x,y,z $
轴正方向的单位向量。向量积的结果仍然是向量。
向量积不满足交换律,即$ \vec{a}\times
\vec{b}\neq \vec{b}\times\vec{a} $,
而是满足$ \vec{a}\times \vec{b}=-\vec{b}
\times\vec{a} $. 

\item 对于二维向量,$ \vec{a}=(a_1,a_2),\ \vec{b}=(b_1,b_2) $,有
%\renewcommand{\arraystretch}{1}
\begin{align}
    |\vec{a}|^2|\vec{b}|^2-|\vec{a}|^2|\vec{b}|^2
    \cos^2\langle\vec{a},\vec{b}\rangle 
    &=|\vec{a}|^2|\vec{b}|^2\sin^2\langle\vec{a},\vec{b}\rangle \nonumber\\
    (a_1^2+a_2^2)(b_1^2+b_2^2)-(a_1b_1+a_2b_2)^2 &=
    (a_1b_2-a_2b_1)^2=
    \begin{vmatrix}
        a_{1} & a_{2} \\
        b_{1} & b_{2}
    \end{vmatrix}
    ^2 \label{三角形面积公式x1y2-x2y1}
\end{align}
所以,$ |\vec{a}||\vec{b}|\sin\langle\vec{a},\vec{b}
\rangle=|a_1b_2-a_2b_1| $,
这正是以$ \vec{a},\vec{b} $为邻边的平行四边形的面积,乘上
$ \dfrac{1}{2} $便能得到以$ \vec{a},\vec{b} $为邻边的三角形
的面积。严格来讲,二维向量没有叉乘,应该表述成$ z $坐标为0的
三维向量的叉乘。某些教辅上可能有此类错误写法
$ |\vec{a}\times\vec{b}|=|(a_1,a_2)\times (b_1,b_2)|=
|a_1b_2-a_2b_1| $,应该修改成
$ |(a_1,a_2,0)\times (b_1,b_2,0)|=|a_1b_2-a_2b_1| $ . 

\item $^*$ (\ref{三角形面积公式x1y2-x2y1})式也可写成
\begin{align} 
    (a_1^2+a_2^2)(b_1^2+b_2^2)=&\  (a_1b_1+a_2b_2)^2+
    (a_1b_2-a_2b_1)^2 \\
    %    \label{向量点乘叉乘平方和} \\
    =&\  (a_1b_1-a_2b_2)^2+(a_1b_2+a_2b_1)^2 
    %    \label{复数实部虚部平方和}
\end{align}
实际上,$ (a_1+a_2 \i)(b_1+b_2 \i)=(a_1b_1-a_2b_2)+(a_1b_2+
a_2b_1)\i $,其中$ \i $是虚数单位。
请读者通过以上两式体会平面向量与复数的联系。
%如果定义向量$ \vec{a},\vec{b} $之间的
%新的数量积为$ \vec{a} \circ \vec{b}=a_1b_1-a_2b_2 $,
%或者$ \vec{a} \circ \vec{b}=a_1b_2+a_2b_1 $,
%请读者思考,这两种新数量积是否满足交换律$ \vec{a}\circ\vec{b}
%= \vec{b} \circ \vec{a} $和分配律
%$ (\vec{a} + \vec{b})\circ \vec{c}=
%\vec{a} \circ \vec{c}+\vec{b} \circ
%\vec{c} $\ ?

\item $^*$ 三维向量满足以下恒等式:
\begin{align*}
    &\ (a_1^2+a_2^2+a_3^2)(b_1^2+b_2^2+b_3^2)\\
    =&\  (a_1b_1+a_2b_2+a_3b_3)^2+
    (a_2b_3-a_3b_2)^2+(a_3b_1-a_1b_3)^2+(a_1b_2-a_2b_1)^2 \\
    =&\ (\vec{a}\cdot \vec{b})^2+(\vec{a}\times\vec{b})^2 \\
    =&\ |\vec{a}|^2|\vec{b}|^2\cos^2\langle\vec{a},\vec{b}\rangle 
    +|\vec{a}|^2|\vec{b}|^2\sin^2\langle\vec{a},\vec{b}\rangle
\end{align*} 
$ \vec{a}\times\vec{b} $同时垂直于$ \vec{a} $和$ \vec{b} $,
方向由右手法则\footnote{
    假设已经将$ \vec{a},\vec{b} $的起点重合。
    伸出右手,(除大拇指外的)四根手指从$ \vec{a} $转向$ \vec{b} $,
    且转过的角度小于$ 180^{\circ} $,此时,竖起的大拇指所指
    的方向就是$ \vec{a}\times\vec{b} $的方向。}确定。
而$ |\vec{a}\times\vec{b}|=|\vec{a}||\vec{b}|\sin\langle\vec{a},
\vec{b}\rangle $. 高中阶段,向量积最大的用处就是在
立体解析几何中计算平面的法向量,以及理解物理学中的
洛伦兹力$ \vec{F}=q\vec{v}\times \vec{B} $和安培力
$ \vec{F}=I\vec{L}\times \vec{B} $. 

以上恒等式其实是$ n $维拉格朗日恒等式的特例:
\begin{gather}\label{拉格朗日恒等式}
    \left(\sum_{k=1}^{n}a_k^2 \right) \left(\sum_{k=1}^{n}b_k^2 \right)=\left( 
    \sum_{k=1}^{n}a_kb_k\right)^2 +\sum_{1\leq k<l\leq n }\left( 
    a_kb_l-a_lb_k\right)^2 \geq 0  
\end{gather}

\item 在拉格朗日恒等式\eqref{拉格朗日恒等式}中令$ n=4 $,那么等号右侧去掉
$ \left( \sum\limits_{k=1}^{n}a_kb_k\right)^2 $后,剩下的
$ \left( a_kb_l-a_lb_k\right)^2 $一共有$ C_n^2=C_4^2=6 $项,并不等于$ n $,
那么能否对这6项进行一些恒等变换,使它们变成地位平等的4项,
从而实现将三维向量积的定义推广到四维呢?答案是做不到。
感兴趣的读者可自行尝试。不过对于四维情形,有另一种恒等式:
\begin{align*}
    &(a_1^2+a_2^2+a_3^2+a_4^2)(b_1^2+b_2^2+b_3^2+b_4^2) \\
    =\ &(a_1b_1-a_2b_2-a_3b_3-a_4b_4)^2+\\
    &(a_1b_2+a_2b_1+a_3b_4-a_4b_3)^2+\\
    &(a_1b_3+a_3b_1-a_2b_4+a_4b_2)^2+\\
    &(a_2b_3-a_3b_2+a_1b_4+a_4b_1)^2
\end{align*}
这一恒等式实际上来源于四元数乘法的定义。
\footnote{ 类似于(\ref{三维向量叉乘})式定义的向量积还能适用于七维,
    而无法推广到其它维度。
参见 \\
https://www.zhihu.com/question/31549458   \\
https://en.wikipedia.org/wiki/Seven-dimensional\_{}cross\_{}product 
%\hspace{4mm} Given the properties of bilinearity, orthogonality and magnitude, a nonzero cross product exists only in three and seven dimensions. This can be shown by postulating the properties required for the cross product, then deducing an equation which is only satisfied when the dimension is 0, 1, 3 or 7. In zero dimensions there is only the zero vector, while in one dimension all vectors are parallel, so in both these cases the product must be identically zero.
%
%\hspace{4mm} The restriction to 0, 1, 3 and 7 dimensions is related to Hurwitz's theorem, that normed division algebras are only possible in 1, 2, 4 and 8 dimensions. The cross product is formed from the product of the normed division algebra by restricting it to the 0, 1, 3, or 7 imaginary dimensions of the algebra, giving nonzero products in only three and seven dimensions. 
%
%\hspace{4mm} In contrast to the three-dimensional cross product, which is unique (apart from sign), there are many possible binary cross products in seven dimensions. 
}

\item 按下面的方法定义的向量乘法适用于任意维度,
\begin{gather}\label{向量逐位乘法}
    \vec{a}*\vec{b}=(a_1a_2,b_1b_2,\cdots,a_nb_n)
\end{gather}
乘出的结果也是同样维数的向量,这种运算法则是使用MATLAB软件所必须理解的。
为了避免和(\ref{三维向量叉乘})式定义的向量积混淆,(\ref{向量逐位乘法})
式定义的向量乘法一般不称为向量积,而是称为逐元乘法
(element-wise multiplication). 

\item 向量不论是点乘还是叉乘,都没有对应的除法,
主要是因为除法的结果有无穷多种,这就失去了除法的意义。
比如$ \vec{a}=(x,3-x),\ \vec{b}=(1,1),\ 
\vec{a}\cdot \vec{b}=3 $,那么$ 3\div \vec{b} $的结果有无穷多种。
又比如$ \vec{a}=(x,x+1,x+1),\ \vec{b}=(1,1,1),\ 
\vec{a}\times\vec{b}=(0,1,-1) $,那么$ (0,1,-1) \div \vec{b} $
的结果有无穷多种。换一种方式描述就是:由$ \vec{a}\divideontimes 
\vec{b}=\vec{c}\divideontimes \vec{b} $不能推出
$ \vec{a}=\vec{c} $\ (其中的$ \divideontimes $代表点乘或叉乘),
即使$ \vec{b} $的每一个坐标分量都不为0.

\end{itemize}

\section{正余弦定理和三角形面积公式}
\begin{itemize}[leftmargin=\inteval{\myitemleftmargin}pt,itemsep=
   \inteval{\myitemitempsep}pt,topsep=\inteval{\myitemtopsep}pt]
\item 正弦定理:$ \dfrac{a}{\sin A}=\dfrac{b}{\sin B}=
\dfrac{c}{\sin C} =2R,\ R $为三角形外接圆半径。

\item 余弦定理:
\begin{gather*}
    c^2=a^2+b^2-2\vec{a}\cdot\vec{b}
    =a^2+b^2-2ab\cos C  \\
    \vec{a}\cdot\vec{b}=\dfrac{a^2+b^2-c^2}{2} 
    \hspace{1.5cm}     \cos C=\dfrac{a^2+b^2-c^2}{2ab}  
\end{gather*}

\item 对任意三角形,“大边”是”大角”的充要条件。分析如下:\\
\mycircled{1}若$ A,B,C\in(0,\dfrac{\pi}{2}] $,根据正弦定理和正弦函数的单调性,必有“大边对大角”、
“大角对大边”。\\
\mycircled{2}若三角形有一个角是钝角,不妨设为$ C $,根据余弦定理,
$ c^2=a^2+b^2-2ab\cos C $,$ c^2 $同时大于$ a^2 $和$ b^2 $,说明$ c $是最长的边。
反过来,若$ c $是最长的边,$ C $一定是钝角。剩余两个锐角符合\mycircled{1}中的情形。

所以,对任意三角形,“大边”是”大角”的充要条件。

事实上,以下4个条件是等价的,任意两者互为充要条件:
\begin{align*}
    a>b \Leftrightarrow A>B \Leftrightarrow \sin A>\sin B \Leftrightarrow 
    |\tan A|>|\tan B| 
\end{align*}
若其中出现直角,把$ \left|\tan \dfrac{\pi}{2}\right| $视为$ +\infty $即可。

\item 如果$ \Delta ABC $三条边长都是有理数,根据余弦定理,三个角的余弦值
都是有理数。于是$ \cos2A=2\cos^2A-1 $也是有理数,$ \cos2B,\cos2C $同样是有理数。
进一步地,若$ n\in \textbf{N}^+ $,容易验证:
\begin{align*}
    \cos(n+1)A =2\cos nA\cos A -\cos(n-1)A
\end{align*}
利用数学归纳法容易证明,$ \cos nA,\ \cos nB,\ \cos nC $都是有理数。
2010年的江苏高考考察了这一知识点。

\item 将三个余弦定理的式子相加可得:$ a^2+b^2+c^2=2bc\cos A+2ac\cos B+
2ab\cos C $,若将此式中的$ a,b,c $换成任意实数$ x,y,z $,
同时保持$ A+B+C=\pi $,那么有
\begin{gather*}
    x^2+y^2+z^2\geq 2yz\cos A+2zx\cos B + 2xy\cos C
\end{gather*}
此不等式被称为嵌入不等式,其证明\footnote{提示:将要证的不等式
    $ x^2+y^2+z^2-2yz\cos A-2zx\cos B -2xy\cos C \geq 0 $
    看成关于$ x $的二次函数,然后进行配方。}参见(\ref{嵌入不等式})。
$ x^2+y^2+z^2\geq xy+yz+zx $是显然的,不过,$ xy+yz+zx $与
$ 2yz\cos A+2zx\cos B + 2xy\cos C $的大小关系是不确定的
(即使限制为锐角三角形,也还是不确定)。
比如$ a=2,\ b=3,\ c=4 $,若$ x=2,\ y=5,\ z=7 $,则
$ 2yz\cos A+2zx\cos B + 2xy\cos C-(xy+yz+zx)=16.5>0 $. 
若$ x=6,\ y=5,\ z=7 $,则
$ 2yz\cos A+2zx\cos B + 2xy\cos C-(xy+yz+zx)=-3<0 $. 

\item 在$ \Delta ABC $中,$ c=a\cos B+b\cos A\geq a\cos A+
b\cos B $,此不等式等价于\\ $ (a-b)(\cos B-\cos A)\geq 0 $,
而根据“大边对大角”,这是显然的。所以有
\begin{align*}
    a \geq &\ b\cos B+ c\cos C  \\
    b \geq &\ a\cos A+ c\cos C  \\
    c \geq &\ a\cos A+ b\cos B  	
\end{align*}
将以上三式相加可得:$ a+b+c=2p \geq 2(a\cos A+b\cos B+ c\cos C) $. 

\item 设平行四边形两条对角线长度分别为$ l_1,l_2 $,两条邻边的长度分别为$ a,b $,
则
\begin{gather*}
    l_1^2+l_2^2=a^2+b^2-2ab\cos\theta+a^2+b^2+2ab\cos\theta=2(a^2+b^2)
\end{gather*}
该结果可改写成
\begin{gather*}
    \left(\dfrac{1}{2}l_1\right)^2=\dfrac{2(a^2+b^2)-l_2^2}{4}
\end{gather*}
此式可用于计算三角形的中线长。 

\item 对于任意四边形$ ABCD $,求证:$ \vec{AC}\perp
\vec{BD} $是$ BC^2+AD^2=BA^2+CD^2 $的充要条件。
\begin{figure}[h]
    \centering
    \includegraphics[width=0.26\linewidth]{对角线垂直的四边形}
    \caption{四边形$ ABCD $}
    \label{图8-1四边形ABCD}
\end{figure} \\
\textbf{方法一}
\begin{align*}
    \vec{BC}\cdot\vec{BD}=\dfrac{BC^2+BD^2-CD^2}{2} \\
    \vec{BA}\cdot\vec{BD}=\dfrac{BA^2+BD^2-AD^2}{2} 
\end{align*}
两式相减:
\begin{align*}
    \vec{AC}\cdot\vec{BD}=\dfrac{BC^2+AD^2-BA^2-CD^2}{2}
\end{align*}
这说明$ \vec{AC}\perp \vec{BD} $是
$ BC^2+AD^2=BA^2+CD^2 $的充要条件。 \\
\textbf{方法二}\ 在空间中任取一点$ O $,由$ BC^2+AD^2=BA^2+CD^2 $可得:
\begin{gather*}
    (\vec{OC}-\vec{OB})^2 +
    (\vec{OD}-\vec{OA})^2 =
    (\vec{OA}-\vec{OB})^2 +
    (\vec{OD}-\vec{OC})^2  \quad \Leftrightarrow 	
\end{gather*}
\vspace{-0.8cm}
\begin{gather}\label{OAOBOCOD}
    \vec{OC}\cdot\vec{OB}+
    \vec{OD}\cdot\vec{OA}=
    \vec{OA}\cdot\vec{OB}+
    \vec{OD}\cdot\vec{OC} \quad \Leftrightarrow
\end{gather}
\vspace{-0.8cm}
\begin{gather*}	 
    (\vec{OC}-\vec{OA})
    \cdot(\vec{OB}-\vec{OD})=0 \quad \Leftrightarrow \\
    \vec{AC}\cdot\vec{DB}=0
\end{gather*}
(\ref{OAOBOCOD})式体现了$ A,B,C,D $四点的地位平等。

细心的读者应该能注意到,本题没有强调$ ABCD $是平面四边形还是空间四边形(三棱锥),
事实上对此没有要求,不论是哪种情形,以上两种证明过程都是成立的。

\item 现在我们把图\ref{图8-1四边形ABCD}看作是三棱锥,
假设其中两组对棱互相垂直,求证:第三组对棱也互相垂直。\\
\textbf{方法一}\ 设$ \vec{AB}\perp \vec{CD},\ 
\vec{AD}\perp \vec{BC} $,在空间中任取一点$ O $,那么
\begin{align*}
    (\vec{OA}-\vec{OB})\cdot
    (\vec{OC}-\vec{OD}) =&\ 0  \\
    (\vec{OA}-\vec{OD})\cdot
    (\vec{OB}-\vec{OC}) =&\ 0 
\end{align*}
由以上两式可得:
\begin{align}\label{三组对棱垂直的连等式}
    \vec{OA}\cdot \vec{OC}+
    \vec{OB}\cdot \vec{OD}=
    \vec{OA}\cdot \vec{OD}+
    \vec{OB}\cdot \vec{OC}=
    \vec{OA}\cdot \vec{OB}+
    \vec{OC}\cdot \vec{OD}
\end{align}
由(\ref{三组对棱垂直的连等式})式可得:$ \vec{AC}\cdot
\vec{DB}=0 $. (\ref{三组对棱垂直的连等式})式与(\ref{OAOBOCOD})式相似,
也与(\ref{垂心性质三个连等式})式相似。\\
\textbf{方法二}\ 设$ A $在平面$ BCD $上的投影的$ H $,
\begin{figure}[h]
    \centering
    \includegraphics[width=0.25\linewidth]{三组对棱垂直}
\end{figure}
\begin{align*}
    AB\perp CD,\ AH\perp CD \Rightarrow BH\perp CD \\
    AD\perp BC,\ AH\perp BC \Rightarrow DH\perp BC 
\end{align*}
所以,$ H $为$ \Delta BCD $的垂心。$ CH\perp DB,\ AH\perp DB \Rightarrow 
AC\perp DB $. 

\item 考虑三角函数的万能代换,$ \sin\alpha=\dfrac{2\tan \frac{x}{2}}{1+\tan^2 \frac{x}{2}},\cos\alpha=\dfrac{1-\tan^2 \frac{x}{2}}{1+\tan^2 \frac{x}{2}} $,
如果令$ \tan\dfrac{x}{2}=\dfrac{b}{a} $,那么
\begin{align*}
    \sin\alpha=\dfrac{\frac{2b}{a}}{1+\left(\frac{b}{a}\right)^2}=
    \dfrac{2ab}{a^2+b^2},\quad\quad \cos\alpha=\dfrac{1-\left(\frac{b}{a}
    \right)^2}{1+\left(\frac{b}{a}\right)^2}=\dfrac{a^2-b^2}{a^2+b^2}
\end{align*}
利用$ \cos^2 \alpha+\sin^2 \alpha=1 $,可得到
\begin{gather}\label{勾股定理正整数解通式}
    (a^2-b^2)^2+(2ab)^2=(a^2+b^2)^2
\end{gather}
令$ x=a^2-b^2,\ y=2ab,\ z=a^2+b^2 $,那么上式变成$ x^2+y^2=z^2 $,
具有勾股定理的形式。$ a^2-b^2 $和$ 2ab $分别是$ (a+b\i)^2=(a^2-b^2)+2ab\i $
的实部和虚部。
让$ a,b (a\neq b) $取正整数,就能得到勾股数,所以,方程$ x^2+y^2=z^2 $
有无穷多组整数解\footnote{费马大定理:当整数
$ n\geq 3 $时,方程$ x^n+y^n=z^n $没有$ xyz\neq 0 $的整数解(也可等价地表述为
没有正整数解)。感兴趣的读者可以探讨:方程$ k_1 x^n+k_2 y^n=k_3 z^n \ 
(k_1,k_2,k_3\in \textbf{Z},\ n\geq 2 ) $ 是否有正整数解。}。
常见的勾股数有 $ (3,4,5),(5,12,13),(7,24,25),(8,15,17),(9,40,41),
(11,60,61),(20,21,29) $. 

若把(\ref{勾股定理正整数解通式})式中的$ a^2,b^2 $换成$ a,b $,有
\begin{align*}
    (a-b)^2+4ab=(a+b)^2
\end{align*}
再把上式的$ a,b $换成向量$ \vec{a},\vec{b} $,然后移项,有
\begin{align}\label{极化恒等式}
    4 \vec{a}\cdot \vec{b}=
    (\vec{a}+\vec{b})^2-(\vec{a}-\vec{b})^2
\end{align}
(\ref{极化恒等式})式被称为极化恒等式。

\item 海伦公式。设$ \Delta ABC $的三条边长分别为$ a,b,c $,
令$ p=\dfrac{a+b+c}{2} $,为三角形周长的一半,
则$ S_{\Delta ABC}=\sqrt{p(p-a)(p-b)(p-c)} $. \ 证明如下:
\begin{align*}
    S_{\Delta ABC}^2=&\ \left(\dfrac{1}{2}ab\sin C \right) ^2=
    \dfrac{1}{4}a^2b^2\left[ 1-\left( {\dfrac{a^2+b^2-c^2}{2ab}}\right)^2 \right] 
\end{align*}
将括号展开后可得:
\begin{align*}
    S_{\Delta ABC}^2=\dfrac{1}{16}\left[-(a^4+b^4+c^4)+2(a^2b^2+b^2c^2+c^2a^2)\right]
\end{align*}
不做展开,而进行因式分解可得:
\begin{align*}
    S_{\Delta ABC}^2
    =&\ \left[ \dfrac{1}{2}ab+\dfrac{1}{4}
    (a^2+b^2-c^2)\right] \left[ \dfrac{1}{2}ab-\dfrac{1}{4} (a^2+b^2-c^2) \right] \\
    =&\ \left[  \dfrac{1}{4}(a+b)^2-\dfrac{1}{4}c^2 \right] 
    \left[ -\dfrac{1}{4}(a-b)^2+\dfrac{1}{4}c^2 \right] \\
    =&\ \left(\dfrac{a+b+c}{2}\right) \left(\dfrac{a+b-c}{2}\right) 
    \left(\dfrac{b+c-a}{2}\right) \left(\dfrac{a+c-b}{2}\right) \\
    =&\ p(p-a)(p-b)(p-c)
\end{align*}
若$ p $是定值(即三角形的周长固定),根据均值不等式,有
\begin{gather}\label{正三角形面积最大}
    p(p-a)(p-b)(p-c)\leq p\left[\dfrac{(p-a)+(p-b)+(p-c)}{3}\right]^3=\dfrac{p^4}{27}
\end{gather}
以上的等号在$ a=b=c $时成立,所以,在周长固定的情况下,等边三角形的面积最大。

记$ u=a^4+b^4+c^4,\ v=a^2b^2+b^2c^2+c^2a^2 ,\ w=a^2+b^2+c^2 $,
那么$ w^2=u+2v $,根据柯西不等式:$ u\geq v $,那么$ w^2=u+2v \geq 3(-u+2v)=
48 \cdot \dfrac{-u+2v}{16}=48 S_{\Delta ABC}^2 $,开平方后有:
$ w\geq 4\sqrt{3}S_{\Delta ABC} $,即
\begin{gather}\label{外森比克不等式}
    a^2+b^2+c^2 \geq 4\sqrt{3}S_{\Delta ABC}
\end{gather}
此不等式被称为外森比克(Weitzenböck)不等式。
等号成立的条件是$ a=b=c $,即$ \Delta ABC $
是等边三角形。再利用余弦定理来逆向分析外森比克不等式,
$ a^2+b^2+c^2=a^2+b^2+(a^2+b^2-2ab\cos C)\geq 4\sqrt{3}\cdot
\dfrac{1}{2}ab\sin C $,移项,
$ 2(a^2+b^2)\geq 2ab(\cos C+\sqrt{3}\sin C)=4ab\sin(C+\dfrac{\pi}{6}) $,
此不等式显然成立。 所以,当$ a^2+b^2+c^2=4\sqrt{3}S_{\Delta ABC}=
\left\{\begin{aligned}
    & 2\sqrt{3}ab\sin C \\
    & 2\sqrt{3}bc\sin A \\
    & 2\sqrt{3}ac\sin B
\end{aligned}\right. $时,$ \Delta ABC $一定是等边三角形。

\item 三角形面积公式汇总:
\begin{gather}\label{三角形面积公式汇总}
    \left.    
    \begin{gathered}
        S=\dfrac{1}{2}ab\sin C=\dfrac{1}{2}bc\sin A=\dfrac{1}{2}ac\sin B=\\ 
        2R^2\sin A\sin B\sin C=\dfrac{abc}{4R}=\sqrt{p(p-a)(p-b)(p-c)}=pr
    \end{gathered} \right\}
\end{gather}
其中,$ R $是外接圆半径,$ r $是内切圆半径,$ p $为半周长。

\item 外森比克不等式的等价形式:
\begin{align*}
    a^2+b^2+c^2 &=2bc\cos A+2ac\cos B+2ab\cos C \\
    &=8R^2(\sin B\sin C\cos A+\sin A\sin C\cos B+\sin A\sin B\cos C) \\
    &\geq 4\sqrt{3}S_{\Delta}=4\cdot 2R^2\sin A\sin B\sin C
\end{align*}
两边约去$ 8R^2 $后可得:
\begin{gather*}
    \sin B\sin C\cos A+\sin A\sin C\cos B+\sin A\sin B\cos C
    \geq \sqrt{3} \sin A\sin B\sin C
\end{gather*}

\item $^*$ 设三角形的三条边长分别为$ a,b,c $,请思考:\\
\mycircled{1} 给定约束条件$ \lambda_1 a+\lambda_2 b+\lambda_3 c=\mu $,
($ \lambda_1,\lambda_2,\lambda_3,\mu $均为正数),三角形的面积何时最大?\\
\mycircled{2} 给定约束条件$ \lambda_1 a^2+\lambda_2 b^2+\lambda_3 c^2=\mu $,
($ \lambda_1,\lambda_2,\lambda_3,\mu $均为正数),三角形的面积何时最大?\\
对于\mycircled{1},容易想到
\begin{align*}
    p(p-a)(p-b)(p-c)=&\ \dfrac{p}{\lambda_1 \lambda_2 \lambda_3}
    (\lambda_1 p-\lambda_1 a)(\lambda_2 p-\lambda_2 b)
    (\lambda_3 p-\lambda_3 c)\\ \leq&\ 
    \left[\dfrac{\left(\frac{1}{\lambda_1
            \lambda_2 \lambda_3}+\lambda_1+\lambda_2+
        \lambda_3\right)p-\mu}{4}\right]^4
\end{align*}
但是,$ p=\dfrac{1}{2}(a+b+c) $不再是常数,而且等号很可能无法成立,
所以上面方法不可行。\\
对于\mycircled{2},利用
\begin{gather*}
    S=\frac{1}{2}ab\sin C=\frac{1}{2}\sqrt{a^2b^2(1-\cos^2 C)}=
    \frac{1}{2}\sqrt{a^2b^2-\frac{1}{4}(a^2+b^2-c^2)^2}  \\
    = \frac{1}{2}\sqrt{a^2b^2-\frac{1}{4}\left(a^2+b^2-
        \dfrac{\mu-\lambda_1 a^2-\lambda_2b^2}{\lambda_3}\right)^2}
\end{gather*}
令$ M=\dfrac{1}{\left(1+\frac{\lambda_1}{\lambda_3}\right)
    \left(1+\frac{\lambda_2}{\lambda_3}\right)} $,
$ u=\left(1+\dfrac{\lambda_1}{\lambda_3}\right)a^2 $,
$ v=\left(1+\dfrac{\lambda_2}{\lambda_3}\right)b^2 $,则
\begin{gather*}
    a^2b^2=Muv \leq M\left(\dfrac{u+v}{2}\right)^2 \\
    S\leq \frac{1}{2}\sqrt{M\left(\dfrac{u+v}{2}\right)^2
        -\frac{1}{4}\left(u+v-\mu \right)^2} 
\end{gather*}
再做变量代换$ t=u+v $,根号下就是关于$ t $的二次函数,剩余略。

\item $ \Delta ABC $的$ BC $边上有$ n $个点,记为$ P_1,P_2,\cdots P_n $,
设$ \vec{AP_1}+\vec{AP_2}+\cdots +\vec{AP_n}=
\lambda \vec{AB} +\mu \vec{AC} $,则$ \lambda+\mu=n $. \begin{figure}[h]
    \centering
    \includegraphics[width=0.3\linewidth]{多个向量终点共线}
\end{figure}

\end{itemize}

\section{重心、垂心、内心、外心}
\begin{itemize}[leftmargin=\inteval{\myitemleftmargin}pt,itemsep=
   \inteval{\myitemitempsep}pt,topsep=\inteval{\myitemtopsep}pt]
\item $ \Delta ABC $的三条中线交于一点$ G $(重心),三条中线
把三角形分成6个面积相等的小三角形。重心分割中线的比例为1:2,证明如下:
\begin{figure}[h]
    \centering
    \includegraphics[width=0.3\linewidth]{重心性质证明}
\end{figure} \\
\textbf{方法一}\ 过$ B $点做$ AC $的平行线,与$ AE $的延长线交于点$ J $,容易看出
$ \Delta AFG \backsim \Delta JBG $(因为三组对应角相等),所以$ \dfrac{BG}{FG}=
\dfrac{JB}{AF}=2 $. \\
\textbf{方法二}\ 过$ E $点做$ CD $的平行线,与$ AB $交于点$ K $,容易看出
$ \Delta BEK \backsim \Delta BCD,\\ \Delta AGD \backsim \Delta AEK $,所以
$ DK=BK=\dfrac{1}{2}AD,\ \dfrac{AG}{GE}=\dfrac{AD}{DK}=2 $. \\
\textbf{方法三}\ 过$ F $点做$ CB $的平行线,与$ AE $交于点$ L $,容易看出
$ \Delta LFG \backsim \Delta EBG $,所以$\dfrac{BG}{FG}=\dfrac{BE}{FL}=2 $. \\
\textbf{方法四}\ 向量法,在$ \Delta ABC $中,$ AB $边的中点为$ D $,
$ BC $边的中点为$ E $,$ CA $边的中点为$ F $. 
\begin{align*}
    & \vec{AG}=\lambda \vec{AD}+(1-\lambda)\vec{AC}=
    \mu \vec{AF}+(1-\mu)\vec{AB} \\
    & \vec{AG}=\dfrac{\lambda}{2} \vec{AB}+(1-\lambda)\vec{AC}=
    \dfrac{\mu}{2} \vec{AC}+(1-\mu)\vec{AB} 
\end{align*}
则 $ \left\{ 
\begin{aligned}
    \dfrac{\lambda}{2} =&\ 1-\mu  \\
    \dfrac{\mu}{2} =&\ 1-\lambda
\end{aligned}
\right. $,解得$ \lambda=\mu=\dfrac{2}{3} $,即$ \vec{AG}=
\dfrac{2}{3}\vec{AD}+\dfrac{1}{3}\vec{AC}=
\dfrac{2}{3}\vec{AF}+\dfrac{1}{3}\vec{AB} $,
已经能说明重心分割中线的比例为1:2. 所以$ \vec{AG}=
2\vec{GE}=\vec{GB}+\vec{GC}, 
\ \vec{GA}+\vec{GB}+\vec{GC}=
\vec{0} $. 再设$ O $为空间中的任意一点,则
\begin{align*}
    \vec{OG}=\vec{OA}+\vec{AG} =&\ 
    \vec{OA}+\dfrac{1}{3}\left(\vec{AB}+
    \vec{AC} \right) \\
    =&\ \vec{OA}+\dfrac{1}{3} \left[\left( \vec{OB} 
    -\vec{OA}\right) + \left( \vec{OC} -\vec{OA}\right)\right]  \\
    =&\ \dfrac{1}{3}\left(\vec{OA}+\vec{OB}+
    \vec{OC} \right) 
\end{align*}

\item 四面体的重心是4个顶点与对面重心连线的交点,重心分连线的比例为3:1,
重心也是与四个顶点连线长度的平方和最小的点。

\item 在$ \Delta ABC $中,$ D $是$ BC $边上一点,且$ \vec{BD}=\mu_1
\vec{BC},\ 0<\mu_1<1 $. $ E $是$ AD $上一点,$ \vec{AE} =
\mu_2\vec{AD},\ 0<\mu_2<1 $. 过$ E $点的直线与$ AB,AC $所在的直线分别交于
$ M,N $两点,设$ \vec{AM}=x\vec{AB},\vec{AN}=
y\vec{AC} $,求证:
$ \dfrac{(1-\mu_1)\mu_2}{x}+\dfrac{\mu_1\mu_2}{y}=1 $. 
\begin{figure}[h]
    \centering
    \includegraphics[width=0.3\linewidth]{系数倒数相加为定值}
\end{figure} \\
\textbf{方法一}\ 向量法
\begin{align*}
    \vec{AE} =\mu_2\vec{AD} =&\ \mu_2\left[
    (1-\mu_1)\vec{AB}+\mu_1\vec{AC}\right]\\
    =&\ (1-\mu_1)\mu_2\vec{AB}+\mu_1\mu_2\vec{AC} \\
    \vec{AE}=\lambda\vec{AM}+(1-\lambda)
    \vec{AN}=&\ \lambda x\vec{AB}+(1-\lambda)y
    \vec{AC}
\end{align*}
$ \left\{ 
\begin{aligned}
    &\lambda x = (1-\mu_1)\mu_2 \\
    &(1-\lambda)y = \mu_1\mu_2
\end{aligned} \right. $. 所以$ \dfrac{(1-\mu_1)\mu_2}{x}+\dfrac{\mu_1\mu_2}{y}
=\lambda+1-\lambda=1 $. \\
当$ E $是三角形的重心时,$ \mu_1=\dfrac{1}{2},\ \mu_2=\dfrac{2}{3},\ 
\dfrac{1}{x}+\dfrac{1}{y}=3 $. \\
\textbf{方法二}\ 面积法。
\begin{gather*}
    \dfrac{S_{\Delta AME}+S_{\Delta ANE}}{S_{\Delta ABC}}=
    \dfrac{S_{\Delta AMN}}{S_{\Delta ABC}}=\dfrac{AM\cdot AN\cdot\sin\angle BAC}
    {AB\cdot AC\cdot\sin\angle BAC}=xy \\
    \dfrac{S_{\Delta AME}}{S_{\Delta ABC}}=\dfrac{S_{\Delta AME}}
    {S_{\Delta ABD}}\cdot \dfrac{S_{\Delta ABD}}{S_{\Delta ABC}}=x\mu_2\cdot\mu_1 \\
    \dfrac{S_{\Delta ANE}}{S_{\Delta ABC}}=y\mu_2\cdot(1-\mu_1)
\end{gather*}
于是有$ x\mu_2\mu_1+y\mu_2(1-\mu_1)=xy $,两边同除$ xy $即得结论。

\item 设$ \Delta ABC $的垂心为$ H $,则$ \vec{HA}\cdot \vec{BC}=
\vec{HA}\cdot (\vec{HC}-\vec{HB})=
\vec{0},\ \vec{HA}\cdot \vec{HC}=
\vec{HA}\cdot \vec{HB} $,采用类似方法,可以得到
\begin{align}\label{垂心性质三个连等式}
    \vec{HA}\cdot \vec{HB}=
    \vec{HA}\cdot \vec{HC}=\vec{HB}\cdot \vec{HC} 
\end{align}
\begin{figure}[h]
    \centering
    \includegraphics[width=0.3\linewidth]{垂心的性质}
\end{figure}
考虑$ \Delta AHB,\Delta BHC,\Delta CHA $的面积比,
\begin{gather*}
    \dfrac{S_{\Delta AHB}}{S_{\Delta BHC}}=\dfrac{HB\cdot AD}
    {HB \cdot CD}=\dfrac{AD}{CD}=\dfrac{\dfrac{BD}{\tan A}}
    {\dfrac{BD}{\tan C}}=\dfrac{\tan C}{\tan A} \\
    S_{\Delta BHC}:S_{\Delta CHA}:S_{\Delta AHB}= \tan A:\tan B:\tan C
\end{gather*}
四边形$ AEHD $有两个内角为直角,所以$ \angle EAD+\angle EHD=\angle EAD
+\angle BHC =\pi $,同理可得:
$ \angle BCA+\angle BHA =\pi,\ \angle ABC+\angle AHC =\pi $. 

\item 设$ \Delta ABC $的内心为$ I $,若$ \vec{AI}=\lambda\vec{AB}+\mu\vec{AC} $,
则$ \lambda:\mu=|\vec{AC}|:|\vec{AB}| $. 
\begin{figure}[h]
    \centering
    \includegraphics[width=0.4\linewidth]{内心外心合并图}
\end{figure}
\\
\textbf{证}\ 过$ I $点作$ AB,AC $的平行线,与$ AB,AC $分别交于$ D,E $两点,
则$ \vec{AD}=\lambda\vec{AB},\ 
\vec{AE}=\mu\vec{AC} $,
因为$ \angle DAI=\angle EAI $,所以平行四边形$ ADIE $是菱形,
$ \lambda|\vec{AB}|=\mu|\vec{AC}| $,即
$ \lambda:\mu=|\vec{AC}|:|\vec{AB}| $. 

\item 设$ \Delta ABC $内切圆半径为$ r $,则
\begin{gather*}
    S_{\Delta ABC}=\dfrac{1}{2}(a+b+c)r=pr=\sqrt{p(p-a)(p-b)(p-c)} \\
    r=\sqrt{\dfrac{(p-a)(p-b)(p-c)}{p}} \\
    S_{\Delta BIC}:S_{\Delta CIA}:S_{\Delta AIB}=a:b:c
\end{gather*}
\item $ J,K,L $为内切圆与三条边的切点,$ IJ=IK=IL=r $,记$ AK=AJ=l_1,\ 
BK=BL=l_2,\ CJ=CL=l_3 $,则
$ l_1\tan\dfrac{A}{2}=l_2\tan\dfrac{B}{2}=l_3\tan\dfrac{C}{2}=r $,
\begin{align*}
    p=l_1+l_2+l_3=&\ r\left(\dfrac{1}{\tan\dfrac{A}{2}}
    +\dfrac{1}{\tan\dfrac{B}{2}}+\dfrac{1}{\tan\dfrac{C}{2}} \right) \\
    =&\ \dfrac{r\left(\tan\dfrac{A}{2}\tan\dfrac{B}{2}+
        \tan\dfrac{A}{2}\tan\dfrac{C}{2}+\tan\dfrac{B}{2}\tan\dfrac{C}{2}
        \right)}{\tan\dfrac{A}{2}\tan\dfrac{B}{2}\tan\dfrac{C}{2}} \\
    =&\ \dfrac{r}{\tan\dfrac{A}{2}\tan\dfrac{B}{2}\tan\dfrac{C}{2}}
\end{align*}
所以$ S_{\Delta ABC}=pr=p^2\tan\dfrac{A}{2}\tan\dfrac{B}{2}\tan\dfrac{C}{2}=
\dfrac{r^2}{\tan\dfrac{A}{2}\tan\dfrac{B}{2}\tan\dfrac{C}{2}} $. \\
\textbf{注}\ 由$ \tan(\dfrac{A}{2}+\dfrac{B}{2})=\dfrac{\tan \dfrac{A}{2}+\tan \dfrac{B}{2}}
{1-\tan \dfrac{A}{2}\tan \dfrac{B}{2}}=\tan(\dfrac{\pi}{2}-\dfrac{C}{2})=
\dfrac{1}{\tan \dfrac{C}{2}} $
可得到$ \tan\dfrac{A}{2}\tan\dfrac{B}{2}+\tan\dfrac{A}{2}\tan\dfrac{C}{2}+
\tan\dfrac{B}{2}\tan\dfrac{C}{2}=1 $. \\
设$ \vec{AI}=\rho\left(\dfrac{\vec{AB}}{|
    \vec{AB}|}+\dfrac{\vec{AC}}{|
    \vec{AC}|}\right) $,那么$ |\vec{AI}|^2=\rho^2(2+2\cos A) $. 
又有$ |\vec{AI}|=\dfrac{r}{\sin \dfrac{A}{2}} $,所以
\begin{gather*}
    |\vec{AI}|^2=\dfrac{r^2}{\sin^2 \dfrac{A}{2}}=
    \dfrac{2r^2}{1-\cos A}= 2\rho^2(1+\cos A) \\
    \rho=\dfrac{r}{\sin A} \\
    \vec{AI}\cdot\vec{BC}=\rho\left(\dfrac{
        \vec{AB}}{|\vec{AB}|}+\dfrac{
        \vec{AC}}{|\vec{AC}|}\right)\cdot(\vec{AC}-
    \vec{AB})=\rho(1+\cos A)\left(|\vec{AC}|
    -|\vec{AB}|\right)=\\
    \dfrac{r(1+\cos A)}{\sin A}
    \left(|\vec{AC}|-|\vec{AB}|\right)=
    \dfrac{r}{\tan \frac{A}{2}}
    \left(|\vec{AC}|-|\vec{AB}|\right)
\end{gather*}

\item 利用正弦定理和(\ref{三角恒等式1})式,
\begin{align*}
    p=\frac{1}{2}(a+b+c)=R(\sin A+\sin B+\sin C)=4R\cos(\dfrac{A}{2})
    \cos(\dfrac{B}{2})\cos(\dfrac{C}{2})
\end{align*}
两边同乘$ 4R\sin(\dfrac{A}{2})\sin(\dfrac{B}{2})\sin(\dfrac{C}{2}) $,有
\begin{align*}
    4pR\sin(\dfrac{A}{2})\sin(\dfrac{B}{2})\sin(\dfrac{C}{2})=
    2R^2\sin A\sin B\sin C=S_{\Delta ABC}=pr
\end{align*}
所以,$ r=4R\sin(\dfrac{A}{2})\sin(\dfrac{B}{2})\sin(\dfrac{C}{2}) $.
设$ I $是$ \Delta ABC $的内心,则
\begin{gather*}
    |AI||BI||CI|=\frac{r^3}{\sin(\dfrac{A}{2})\sin(\dfrac{B}{2})
        \sin(\dfrac{C}{2})}=4Rr^2
\end{gather*}

\item 设$ \Delta ABC $的外心为$ O $,则$ \vec{AO} \cdot \vec{BC}=
\vec{AO}\cdot (\vec{AC}-\vec{AB})=
\dfrac{1}{2}|\vec{AC}|^2-\dfrac{1}{2}|\vec{AB}|^2. $ 
设$ \vec{AO}=\eta\vec{AB}+\kappa\vec{AC} $,那么 
\begin{gather*}
\begin{cases}
    \vec{AO}\cdot\vec{AB}=\eta |AB|^2+\kappa |AB|
    \cdot |AC|\cdot\cos A=\dfrac{1}{2}|AB|^2 \\
    \vec{AO}\cdot\vec{AC}=\eta |AB|\cdot |AC|\cdot\cos A+\kappa 
    |AC|^2=\dfrac{1}{2}|AC|^2 
\end{cases} \\
\begin{cases}
    \eta |AB|+\kappa |AC|\cdot\cos A=\dfrac{1}{2}|AB| \\
    \eta |AB|\cdot \cos A+\kappa |AC|=\dfrac{1}{2}|AC| 
\end{cases}\q \Rightarrow \q
\begin{cases}
    \eta=\dfrac{|AB|-|AC|\cos A}{2|AB|\sin^2 A} \\
    \kappa=\dfrac{|AC|-|AB|\cos A}{2|AC|\sin^2 A} 
\end{cases}
\end{gather*}
\begin{align*}
    S_{\Delta BOC}:S_{\Delta COA}:S_{\Delta AOB} =&\ \dfrac{1}{2}R^2 
    \sin\angle BOC:\dfrac{1}{2}R^2\sin\angle COA:
    \dfrac{1}{2}R^2\sin\angle AOB \\ =&\ \sin 2A:\sin 2B:\sin 2C
\end{align*}
\item 设$ \Delta ABC $外接圆半径为$ R $,则$ R=\dfrac{abc}{4S} $. 
又有$ a=|BC|=l_2+l_3,b=|AC|=l_1+l_3,c=|AB|=l_1+l_2 $,
$ p-a=l_1,p-b=l_2,p-c=l_3 $,那么
\begin{align*}
    \dfrac{R}{r}=\dfrac{abc}{4S}\cdot \dfrac{p}{S}
    =&\ \dfrac{abc}{4(p-a)(p-b)(p-c)} \\
    =&\ \dfrac{(l_2+l_3)(l_1+l_3)(l_1+l_2)}{4l_1l_2l_3} \\
    \geq &\ \dfrac{2\sqrt{l_2l_3}\cdot 2\sqrt{l_1l_3}\cdot 2\sqrt{l_1l_2}}{4l_1l_2l_3}=2
\end{align*}
上式说明任意三角形的外接圆半径大于等于2倍内切圆半径。当$ l_1=l_2=l_3 $时,
等号成立,此时$ \Delta ABC $是等边三角形。

\item $^*$ 三角形的内心和外心的距离的平方为$ R(R-2r) $,由此也可得出$ R\geq 2r $.
假设一个半径为$ r $的小圆位于一个半径为$ R $大圆的内部(两个圆没有交点),
若两个圆的圆心距恰好等于$ \sqrt{R(R-2r)} $,那么存在无穷多个三角形,
分别以这两个圆为内切圆和外接圆(彭赛列闭合定理)。

\item $ \Delta ABC $内部有任意一点$ O $,记$ \Delta AOB,\Delta BOC,\Delta COA $
的面积分别为$ S_C,S_A,S_B $,那么有:
\begin{align}\label{任意点面积定理}
    S_A\cdot\vec{OA}+ S_B\cdot\vec{OB}
    +S_C\cdot\vec{OC}= \vec{0}
\end{align}
\begin{figure}[h]
    \centering
    \includegraphics[width=0.3\linewidth]{内部点的向量和定理}
\end{figure}
延长$ AO $交$ BC $边于$ D $点,$ \dfrac{S_C}{S_B}=\dfrac{\frac{1}{2}AO\cdot BD \sin \angle  BDO}{\frac{1}{2}AO\cdot DC \sin \angle CDO}=\dfrac{BD}{DC} $,
\begin{gather*}
    \vec{OD}=\dfrac{DC}{BC}\cdot\vec{OB} + 
    \dfrac{BD}{BC}\cdot\vec{OC}=\dfrac{S_B}{S_B+S_C}\cdot
    \vec{OB} +\dfrac{S_C}{S_B+S_C}\cdot\vec{OC} \\
    \dfrac{OD}{OA}=\dfrac{S_{\Delta BOD}}{S_{\Delta BOA}}=
    \dfrac{S_{\Delta COD}}{S_{\Delta COA}}=\dfrac{S_{\Delta BOD}+
        S_{\Delta COD}}{S_{\Delta BOA}+S_{\Delta COA}}=\dfrac{S_A}{S_B+S_C} \\
    \vec{OD}=-\dfrac{S_A}{S_B+S_C}\cdot\vec{OA}=
    \dfrac{S_B}{S_B+S_C}\cdot
    \vec{OB} +\dfrac{S_C}{S_B+S_C}\cdot\vec{OC} \\
    S_A\cdot\vec{OA}+ S_B\cdot\vec{OB}
    +S_C\cdot\vec{OC}= \vec{0}
\end{gather*}
$\diamond$ 当$ O $是$ \Delta ABC $的重心时,
\begin{gather*}
    S_A:S_B:S_C=1:1:1\\  \vec{OA}
    +\vec{OB}+\vec{OC}=\vec{0}
\end{gather*}
$\diamond$ 当$ O $是$ \Delta ABC $的垂心时,
\begin{gather*}
    S_A:S_B:S_C=\tan A:\tan B:\tan C \\ \tan A\cdot
    \vec{OA}+\tan B\cdot \vec{OB}
    +\tan C\cdot\vec{OC}=\vec{0}
\end{gather*}
$\diamond$ 当$ O $是$ \Delta ABC $的内心时,
\begin{gather*}
    S_A:S_B:S_C=a:b:c \\ a\cdot \vec{OA}
    +b\cdot\vec{OB}+
    c\cdot\vec{OC}=\vec{0}
\end{gather*}
$\diamond$ 当$ O $是$ \Delta ABC $的外心时,
\begin{gather*}
    S_A:S_B:S_C=\sin 2A:\sin 2B:\sin 2C  \\
    \sin 2A\cdot\vec{OA}+ \sin 2B\cdot
    \vec{OB}+
    \sin 2C\cdot\vec{OC}=\vec{0}
\end{gather*}

\item (\ref{任意点面积定理})式可以推广到四面体中:设四面体$ A-BCD $内部有任意一点
$ O $,记$  O-ABC,O-ABD,O-ACD,O-BCD $的体积分别为$V_D,V_C,V_B,V_A $,那么有:
\begin{align*}
    V_A\cdot\vec{OA}+ V_B\cdot\vec{OB}
    +V_C\cdot\vec{OC}+V_D\cdot\vec{OD}= \vec{0}
\end{align*}

\item 设$ \Delta ABC $三个顶点的坐标分别为$ A(x_1,y_1),B(x_2,y_2),C(x_3,y_3) $,
平面上有一个动点$ P(x,y) $,考虑$ \lambda_1|PA|^2+\lambda_2|PB|^2+
\lambda_3|PC|^2 $的最小值,($ \lambda_1,\ \lambda_2,\ \lambda_3>0 $),
\begin{align*}
    &\ \lambda_1|PA|^2+\lambda_2|PB|^2+\lambda_3|PC|^2 \\
    =&\ (\lambda_1+\lambda_2+\lambda_3)x^2-2(\lambda_1x_1+\lambda_2x_2
    +\lambda_3x_3)x+\lambda_1x_1^2+\lambda_2x_2^2+\lambda_3x_3^2 \\ 
    +&\ (\lambda_1+\lambda_2+\lambda_3)y^2
    -2(\lambda_1y_1+\lambda_2y_2+\lambda_3y_3)y+
    \lambda_1y_1^2+\lambda_2y_2^2+\lambda_3y_3^2 \\
    =&\ (\lambda_1+\lambda_2+\lambda_3)\left(x-\dfrac{\lambda_1x_1+
        \lambda_2x_2+\lambda_3x_3}{\lambda_1
        +\lambda_2+\lambda_3}\right)^2+\cdots \\
    +&\ (\lambda_1+\lambda_2+\lambda_3)\left(y-\dfrac{\lambda_1y_1
        +\lambda_2y_2+\lambda_3y_3}{
        \lambda_1+\lambda_2+\lambda_3}\right)^2+ \cdots
\end{align*}
$ x,y $是两个完全独立的变量,所以,当$ x=\dfrac{\lambda_1x_1+\lambda_2x_2+
    \lambda_3x_3}{\lambda_1+\lambda_2+\lambda_3} $,$ y=\dfrac{\lambda_1y_1+
    \lambda_2y_2+\lambda_3y_3}{\lambda_1+\lambda_2+\lambda_3} $时,
$ \lambda_1|PA|^2+\lambda_2|PB|^2+\lambda_3|PC|^2 $取得极小值。\\
\mycircled{1} 当$ \lambda_1=\lambda_2=\lambda_3=1 $时,
\begin{gather*}
    x=\dfrac{x_1+x_2+x_3}{3},\q y=\dfrac{y_1+y_2+y_3}{3}
\end{gather*}
$ P $点是$ \Delta ABC $的重心。\\
\mycircled{2} 当$ \lambda_1=\tan A,\ \lambda_2=\tan B,\ \lambda_3=
\tan C $时,
\begin{align*}
    x &=\dfrac{(\tan A)x_1+(\tan B)x_2+(\tan C)x_3}{\tan A+\tan B+
        \tan C} \\ 
    y &=\dfrac{(\tan A)y_1+(\tan B)y_2+(\tan C)y_3}{\tan A+\tan B+
        \tan C}
\end{align*}
$ P $点是$ \Delta ABC $的垂心。\\
\mycircled{3} 当$ \lambda_1=\sin A,\ \lambda_2=\sin B,\ \lambda_3=
\sin C $时,
\begin{align*}
    x &=\dfrac{(\sin A)x_1+(\sin B)x_2+(\sin C)x_3}{\sin A+\sin B+
        \sin C}=\dfrac{ax_1+bx_2+cx_3}{a+b+c} \\ 
    y &=\dfrac{(\sin A)y_1+(\sin B)y_2+(\sin C)y_3}{\sin A+\sin B+
        \sin C}=\dfrac{ay_1+by_2+cy_3}{a+b+c}
\end{align*}
$ P $点是$ \Delta ABC $的内心。\\
\mycircled{4} 当$ \lambda_1=\sin 2A,\ \lambda_2=\sin 2B,\ \lambda_3=
\sin 2C $时,
\begin{align*}
    x &=\dfrac{(\sin 2A)x_1+(\sin 2B)x_2+(\sin 2C)x_3}{\sin 2A+
        \sin 2B+\sin 2C} \\
    y &=\dfrac{(\sin 2A)y_1+(\sin 2B)y_2+(\sin 2C)y_3}{\sin 2A+
        \sin 2B+\sin 2C}
\end{align*}
$ P $点是$ \Delta ABC $的外心。

\item $^*$ 设$ P $是$ \Delta ABC $内的一点,从$ P $点向三条边分别作垂线,
垂足分别为$ D,E,F $,则有Erdos-Mordell不等式\footnote{证明参见
https://forumgeom.fau.edu/FG2007volume7/FG200711.pdf }:
\begin{gather*}
    |PA|+|PB|+|PC|\geq 2(|PD|+|PE|+|PF|)
\end{gather*}
当$ P $是$ \Delta ABC $的外心时,等号成立。

\item 对于$ \Delta ABC $,分别以$ AB,BC,AC $为边,
在$ \Delta ABC $外部作三个等边三角形,
这三个等边三角形的外接圆会交于同一点,该交点被称为\textbf{费马点}
(也称\textbf{托里拆利点})。
\begin{itemize}[itemsep=-3pt]
\item 如果三角形的最大内角小于$ 120^{\circ} $,那么费马点位于
$ \Delta ABC $内部,且费马点(记为$ P $点)就是使到三角形三个顶点距离之和
取最小值的点,$ PA,PB,PC $恰好互成$ 120^{\circ} $.
(证明见本书第\pageref{费马点求偏导}页);
\item 如果三角形的最大内角等于$ 120^{\circ} $,则费马点位于钝角顶点;
\item 如果三角形的最大内角大于$ 120^{\circ} $,则费马点位于$ \Delta ABC $
外部;
\end{itemize}
对于后两种情形(最大内角大于等于$ 120^{\circ} $),钝角顶点是到三个顶点
距离之和最小的点。有些参考资料中把最后一种情形下的钝角顶点
(而不是外接圆交点)称为费马点,这是不恰当的。
\begin{figure}[!h]
    \centering
    \includegraphics[width=0.8\linewidth]{费马点-朝外作三角形}
    \label{费马点朝内朝外两种}
\end{figure}

如果三个等边三角形不是全部在$ \Delta ABC $外部,而是全部在与
$ \Delta ABC $有相交区域的方位(或者说向内部作等边三角形),
三个外接圆也会交于同一点,且交点位于$ \Delta ABC $外部,
此时的交点不具备与三个顶点距离之和最小的特点。
\begin{figure}[!h]
    \centering
    \includegraphics[width=0.8\linewidth]{费马点-朝内作三角形}
\end{figure}
\newpage

\end{itemize}

\section{一些平面几何的定理}
\begin{itemize}[leftmargin=\inteval{\myitemleftmargin}pt,itemsep=
   \inteval{\myitemitempsep}pt,topsep=\inteval{\myitemtopsep}pt]
\item 三角形的内角平分线定理:$ \dfrac{|AB|}{|AC|}=
\dfrac{|BD|}{|CD|} $的充分必要条件是:
$ AD $是$ \angle BAC $的角平分线。(提示,$ |DE|=|DF| $,考虑
$ \Delta ABD $与$ \Delta ACD $的面积之比。)
\begin{figure}[!h]
    \centering
    \includegraphics[width=0.3\linewidth]{三角形的内角平分线定理}
\end{figure} 

\item 梅涅劳斯定理$ ^* $:直线$ l $与$ \Delta ABC $的三边所在直线
分别交于$ D,E,F $点,则有
\begin{gather*}
    \dfrac{|AD|}{|DB|} \cdot \dfrac{|BE|}{|EC|} \cdot
    \dfrac{|CF|}{|FA|}=1
\end{gather*}
\begin{figure}[h]
    \centering
    \includegraphics[width=0.5\linewidth]{梅涅劳斯定理}
\end{figure} \\
\textbf{方法一}\ 
\begin{gather*}
    \dfrac{S_{\Delta ADF}}{S_{\Delta BDE}}=\dfrac{|AD|\cdot|DF|}{|DB|\cdot|DE|},\quad
    \dfrac{S_{\Delta BDE}}{S_{\Delta CFE}}=\dfrac{|BE|\cdot|DE|}{|EC|\cdot|EF|},\quad
    \dfrac{S_{\Delta CFE}}{S_{\Delta ADF}}=\dfrac{|CF|\cdot|EF|}{|FA|\cdot|DF|} \\
    \dfrac{S_{\Delta ADF}}{S_{\Delta BDE}}\cdot\dfrac{S_{\Delta BDE}}{S_{\Delta CFE}}
    \cdot\dfrac{S_{\Delta CFE}}{S_{\Delta ADF}}=\dfrac{|AD|}{|DB|} \cdot
    \dfrac{|BE|}{|EC|} \cdot \dfrac{|CF|}{|FA|}=1
\end{gather*}
\textbf{方法二}\ 分别从$ A,B,C $向$ l $作垂线,垂线段的长度依次为$ r,s,t $,则
\begin{align*}
    \dfrac{|AD|}{|DB|} \cdot	\dfrac{|BE|}{|EC|} \cdot \dfrac{|CF|}{|FA|}=
    \dfrac{r}{s} \cdot \dfrac{s}{t} \cdot \dfrac{t}{r}= 1
\end{align*}
\textbf{方法三}\ 设$ \vec{AD}=k_1\vec{AB},\ \vec{AF}=
k_2\vec{AC},\ k_1\neq k_2 $,因为$ D,F,E $三点共线,那么
\begin{gather*}
    \vec{AE}=\lambda \vec{AD}+(1-\lambda)\vec{AF}=
    \lambda k_1\vec{AB}+(1-\lambda)k_2\vec{AC}
\end{gather*}
又因为$ B,C,E $三点共线,所以$ \lambda k_1+(1-\lambda)k_2=1,\ \lambda=\dfrac{1-k_2}
{k_1-k_2} $,$ \dfrac{|BE|}{|EC|} =-\dfrac{(1-\lambda)k_2}{\lambda k_1} =
\dfrac{k_2(1-k_1)}{k_1(1-k_2)} $,
\begin{align*}
    \dfrac{|AD|}{|DB|} \cdot	\dfrac{|BE|}{|EC|} \cdot \dfrac{|CF|}{|FA|}=
    \dfrac{k_1}{1-k_1}\cdot \dfrac{k_2(1-k_1)}{k_1(1-k_2)}\cdot\dfrac{1-k_2}{k_2}=1
\end{align*}

\item 塞瓦定理$ ^* $:$ \Delta ABC $内部任取一点$ P $,连接$ AP $并延长,交$ BC $
于$ E $点,连接$ BP $并延长,交$ CA $于$ F $点,连接$ CP $并延长,交$ AB $于$ D $点,
则有$ \dfrac{|AD|}{|DB|} \cdot \dfrac{|BE|}{|EC|} \cdot\dfrac{|CF|}{|FA|}=1 $. 
\begin{figure}[H]
    \centering
    \includegraphics[width=0.5\linewidth]{塞瓦定理}
\end{figure}
\textbf{方法一}\ 设$ \Delta APD,\Delta BPD,\Delta BPE,\Delta CPE,
\Delta CPF,\Delta APF $的面积分别为$S_1$,$S_2$,$S_3$,
$S_4$,$S_5$,$S_6$,则
\begin{gather*}
    \dfrac{|BE|}{|EC|}=\dfrac{S_3}{S_4}=\dfrac{S_1+S_2+S_3}{S_4+S_5+S_6}=
    \dfrac{S_1+S_2}{S_5+S_6}		\\
    \dfrac{|AD|}{|DB|} \cdot \dfrac{|BE|}{|EC|} \cdot\dfrac{|CF|}{|FA|}=
    \dfrac{S_5+S_6}{S_3+S_4} \cdot \dfrac{S_1+S_2}{S_5+S_6} \cdot \dfrac{S_3+S_4}{S_1+S_2} =1
\end{gather*} 
\textbf{方法二}\ 设$ \vec{AD}=k_1\vec{AB},\ \vec{AF}=
k_2\vec{AC} $,那么
\begin{align*}
    \vec{AP}=&\ \lambda \vec{AD}+(1-\lambda)\vec{AC}=
    \mu \vec{AF}+(1-\mu)\vec{AB} \\
    =&\ \lambda k_1\vec{AB}+(1-\lambda)\vec{AC}=
    \mu k_2\vec{AC}+(1-\mu)\vec{AB}
\end{align*}
所以,$ \lambda k_1=1-\mu,\ 1-\lambda=\mu k_2 $,解得:$ \lambda=\dfrac{1-k_2}{1-k_1k_2},
\ \mu=\dfrac{1-k_1}{1-k_1k_2} $,$ \dfrac{|BE|}{|EC|} =\dfrac{1-\lambda}{\lambda k_1} =
\dfrac{k_2(1-k_1)}{k_1(1-k_2)} $,
于是
\begin{align*}
    \dfrac{|AD|}{|DB|} \cdot \dfrac{|BE|}{|EC|} \cdot\dfrac{|CF|}{|FA|}=
    \dfrac{k_1}{1-k_1}\cdot \dfrac{k_2(1-k_1)}{k_1(1-k_2)}\cdot\dfrac{1-k_2}{k_2}=1
\end{align*}

\item 托勒密定理$ ^* $:圆的内接四边形两组对边乘积的和等于两条对角线的乘积。\\
\textbf{方法一}\ 圆的半径不影响本题结论的成立与否。设单位圆上不同的四点的坐标为 \\
$ (\cos \alpha_k,\sin \alpha_k)(k=1,2,3,4) $,适当的坐标系选取可保证
$ 0< \alpha_1 <\alpha_2<\alpha_3<\alpha_4<2\pi $,只需验证下式:
\begin{align*}
    \sin\dfrac{\alpha_2-\alpha_1}{2}\cdot 
    \sin\dfrac{\alpha_4-\alpha_3}{2}+\sin\dfrac{\alpha_4-\alpha_1}{2}\cdot
    \sin\dfrac{\alpha_3-\alpha_2}{2}=\sin\dfrac{\alpha_3-\alpha_1}{2}\cdot 
    \sin\dfrac{\alpha_4-\alpha_2}{2}
\end{align*}
\textbf{方法二}\ 设四边形的四个顶点对应的复数分别为$ z_1,z_2,z_3,z_4 $,有恒等式
\begin{align*}
    (z_1-z_2)(z_3-z_4)+(z_1-z_4)(z_2-z_3)=(z_1-z_3)(z_2-z_4)
\end{align*}
在这个等式两边取模,有
\begin{align*}
    |(z_1-z_2)(z_3-z_4)|+|(z_1-z_4)(z_2-z_3)|\geq &\  \\
    |(z_1-z_2)(z_3-z_4)+(z_1-z_4)(z_2-z_3)|=&\ |(z_1-z_3)(z_2-z_4)|
\end{align*}
等号成立的条件是四点共圆或者四点共线。

\end{itemize}

\section{例题}

\begin{enumerate}[label={【\textbf{例\thechapter.\arabic*}】},
 leftmargin=\inteval{\myenumleftmargin}pt,
 itemsep=\inteval{\myenumitempsep}pt,
 itemindent=\inteval{\myenumitemindent}pt]
\item 设$ \Delta ABC $的垂心为$ H $,且$ 3\vec{HA}+4\vec{HB}
+5\vec{HC}=\vec{0} $,求$ \cos\angle BHC $. \\
\textbf{解}\ 根据(\ref{垂心性质三个连等式})式,可设 $ \vec{HA}\cdot
\vec{HB}=\vec{HA}
\cdot\vec{HC}=\vec{HB}\cdot \vec{HC}=t<0 $,分别用
$ \vec{HB},\vec{HC} $与$ 3\vec{HA}+
4\vec{HB} +5\vec{HC}=\vec{0} $做数量积,有
\begin{gather*}
    \begin{cases}
        \vec{HB}\cdot (3\vec{HA}+4\vec{HB} +5\vec{HC})=
        3t+4|\vec{HB}|^2+5t= 0 \\
        \vec{HC}\cdot (3\vec{HA}+4\vec{HB} +5\vec{HC})=
        3t+4t+5|\vec{HC}|^2= 0 
    \end{cases} \Rightarrow \q 
    \begin{cases}
        |\vec{HB}|=\sqrt{-2t}\\
        |\vec{HC}|=\sqrt{-\dfrac{7t}{5}}
    \end{cases}
\end{gather*}
所以,
\begin{gather*}
    \cos\angle BHC=\dfrac{\vec{HB}\cdot \vec{HC}}
    {|\vec{HB}||\vec{HC}|}=\dfrac{t}{\sqrt{-2t}\sqrt{-7t/5}}
    =-\dfrac{\sqrt{70}}{14} 
\end{gather*}

类似地可以计算出$ \cos\angle BHA,\ \cos\angle CHA $,实际上就知道了
$ \cos\angle ABC,\cos\angle BCA,\\ \cos\angle CAB $,
三角形的形状也能确定了。另外,条件$ 3\vec{HA}+4\vec{HB}
+5\vec{HC}=\vec{0} $还可变形成$ 3\vec{HA}+4(\vec{HA}+\vec{AB})
+5(\vec{HA}+\vec{AC})=\vec{0} $,即
$ \vec{AH}=\dfrac{1}{3}\vec{AB}+\dfrac{5}{12}\vec{AC}$.

\item 已知在$ \Delta ABC $中,$ M $是$ BC $中点,$ AM=3,\ BC=10 $,
求$ \vec{AB}\cdot \vec{AC} $的值。\\
\textbf{方法一}\ 由极化恒等式(\ref{极化恒等式}),$ \vec{AB}\cdot 
\vec{AC}=\dfrac{1}{4}[(\vec{AB}+\vec{AC})^2-
(\vec{AB}-\vec{AC})^2]=\dfrac{1}{4}[(2\vec{AM})^2-
(\vec{CB})^2]=(\vec{AM})^2-(\vec{MB})^2=9-25=-16 $. \\
\textbf{方法二}\ 既然$ \vec{AB}\cdot \vec{AC} $的值与三角形形状无关,
不妨将三角形取为特殊形状,比如令$ |AB|=|AC|=\sqrt{3^2+5^2}=\sqrt{34} $,
$ \cos\angle BAC=2\cos^2\angle MAC-1=2\cdot \dfrac{9}{34}-1=-\dfrac{16}{34} $,
$ \vec{AB}\cdot \vec{AC}=\sqrt{34}\cdot \sqrt{34}\cdot
\Big(-\dfrac{16}{34}\Big)=-16 $.

\item 在下图所示的正方形、正六边形和圆形边界上有一点$ P $,
求$ \vec{PA} \cdot \vec{PB} $的范围。
\begin{figure}[h]
    \centering
    \includegraphics[width=0.6\linewidth]{向量点乘最值问题}
\end{figure} \\
\textbf{分析}\ 取$ AB $的中点$ E $,则$ \vec{PA}\cdot\vec{PB}
= \left(\vec{PE}+\vec{EA}\right)\cdot \left(
\vec{PE}-\vec{EA}\right) = \vec{PE}^2-
\vec{EA}^2 $,问题就转化成了求$ |\vec{PE}| $的范围。

另一类问题是$ \vec{PA}\cdot\vec{PB} $为定值
$ L $,求$ P $点的轨迹。只要$ L>-|\vec{EA}|^2 $,$ P $点的轨迹就是以
$ E $点为圆心,以$ \sqrt{|\vec{EA}|^2+L} $为半径的圆。

\item 在$ \Delta ABC $中,$ A=\dfrac{\pi}{3},\ AB=2 $,求
$ \vec{CA}\cdot\vec{CB} $的取值范围。\\
\textbf{解}\ 
$ \vec{CA}\cdot\vec{CB}=\vec{CA}\cdot(
\vec{CA}+\vec{AB})=\vec{CA}^2+
\vec{CA}\cdot\vec{AB}
=|\vec{CA}|^2-|\vec{CA}|\geq -\dfrac{1}{4} $. 

\item $ \Delta ABC $的面积为3,且有$ c=3b $,求$ a $的最小值。\\
\textbf{解}\ $ S_{\Delta ABC}=\dfrac{1}{2}bc\sin A=3,\ b^2=\dfrac{2}{\sin A} $,
$ a^2=b^2+c^2-2bc\cos A=10b^2-6b^2\cos A=\dfrac{4(5-3\cos A)}{\sin A} $,
令$ k=\dfrac{5-3\cos A}{\sin A}>0 $,则$ k\sin A+3\cos A=\sqrt{k^2+9}\sin(A+\varphi)=5 
\Rightarrow \sqrt{k^2+9}\geq 5, k\geq 4 $,所以$ a=2\sqrt{k}\geq 4 $. 

\item 在$ \Delta ABC $中,$ AB=4,\ AC=\sqrt{3}BC $,
求$ \Delta ABC $面积的最大值。\\
\textbf{方法一}\ 以$ AB $所在直线为$ x $轴,$ AB $中点为原点,建立直角坐标系,设
$ A(-2,0),B(2,0),\\ C(x,y) $,那么$ (x+2)^2+y^2=3[(x-2)^2+y^2] \Rightarrow
(x-4)^2+y^2=12 $,说明$ C $点轨迹是一个圆(名为“阿波罗尼奥斯圆”),$ |y|\leq 
2\sqrt{3} $,面积最大值为$ 4\times 2\sqrt{3} \times \dfrac{1}{2}=4\sqrt{3} $. \\
\textbf{方法二}\ $ AC=\sqrt{3}BC=\sqrt{3}a $,
\begin{gather*}
    S=\dfrac{1}{2}\cdot 4 a\sin B=2a\sqrt{1-\cos^2B}=2a\sqrt{1-\left(
        \dfrac{16+a^2-3a^2}{2\cdot4\cdot a}\right)^2} \\
    =\dfrac{1}{2}\sqrt{-(a^2-16)^2+192}\leq 4\sqrt{3}
\end{gather*}

\item 在$ \Delta ABC $中,$ C=\dfrac{2\pi}{3},\ c^2=7b^2,\ S_{\Delta ABC}=
2\sqrt{3} $,求$ b $. \\
\textbf{方法一}\ $ 7b^2=c^2=a^2+b^2-2ab\cos C \Rightarrow a^2+ab-6b^2=0,\ (a+3b)
(a-2b)=0\Rightarrow a=2b,\ S_{\Delta ABC}=\dfrac{1}{2}ab\sin C=\dfrac{1}{2}(2b)b\dfrac{\sqrt{3}}{2}=2\sqrt{3},\ b=2  $. \\
\textbf{方法二}\ $ \sin C=\dfrac{\sqrt{3}}{2},\ \cos C=-\dfrac{1}{2},\ \sin B=
\dfrac{1}{\sqrt{7}}\sin C=\dfrac{\sqrt{21}}{14},\ \cos B=\dfrac{5\sqrt{7}}{14},
\ \sin A=\sin(B+C)=\dfrac{\sqrt{21}}{7},\ \ S_{\Delta ABC}=\dfrac{1}{2}bc\sin A=
\dfrac{1}{2}b(\sqrt{7}b)\dfrac{\sqrt{21}}{7}=2\sqrt{3},\ b=2 $. 

\item (2017 高考全国卷)如下图,在矩形$ ABCD $中,$ AB=1,\ AD=2 $,
动点$ P $在以点$ C $为圆心且与$ BD $相切的圆上运动,若$ \vec{AP}=
\lambda\vec{AB}+\mu\vec{AD} $,
求$ \lambda+\mu $的最大值。 
\begin{figure}[H]
    \centering
    \includegraphics[width=0.35\linewidth]{向量等和线问题(带辅助线)}
\end{figure} 
\noindent\textbf{解}\ 选定一条与$ BD $平行的直线$ l $,当$ P $点在$ l $上运动时,
$ \lambda+\mu $为定值。作一条与$ BD $平行且与$ \odot C $
相切的直线,切点为$ Q $,当$ P $运动到$ Q $点时,$ \lambda+\mu $取得最大值。
根据相似三角形,最大值为3. 

\item 已知向量$ \vec{a},\vec{b} $满足
$ |\vec{a}|=1,|\vec{b}|=2$,
求$ |\vec{a}+\vec{b}|+|\vec{a}-
\vec{b}| $的最小值和最大值。\\
\textbf{解}\ 
\begin{gather*}
    \sqrt{a^2+b^2+2ab\cos\theta}+\sqrt{a^2+b^2-2ab\cos\theta}=
    \sqrt{5+4\cos\theta}+\sqrt{5-4\cos\theta}=\\
    \sqrt{\left(\sqrt{5+4\cos\theta}+\sqrt{5-4\cos\theta} 
        \right)^2}=\sqrt{10+2\sqrt{25-16\cos^2\theta}}
\end{gather*}
最小值是4,此时$ \vec{a},\vec{b} $反向;
最大值是$ 2\sqrt{5} $,此时$ \vec{a},\vec{b} $同向。
还可以求导,或者使用柯西不等式求最大值。

\item 设$ x>0,y>0 $,向量$ \vec{a}=(1-x,4),\vec{b}=
(x,-y) $,求$ \vec{a}//\vec{b} $,则$ x+y $的最小值。\\
\textbf{解}\ $ 4x+(1-x)y=0,\ \dfrac{1}{x}+\dfrac{4}{y}=1 $. \\
\textbf{方法一}\ 消元:
$ x=\dfrac{y}{y-4},x+y=\dfrac{y}{y-4}+y=(y-4)+\dfrac{4}{y-4}+5\geq 9 $. \\
\textbf{方法二}\ 基本不等式:
$ x+y=(x+y)(\dfrac{1}{x}+\dfrac{4}{y})=5+\dfrac{y}{x}+\dfrac{4x}{y}
\geq 5+ 2\sqrt{\dfrac{y}{x}\cdot \dfrac{4x}{y}}=9 $. \\
\textbf{方法三}\ 柯西不等式:$ x+y=(x+y)(\dfrac{1}{x}+\dfrac{4}{y})
\geq \left( \sqrt{x\cdot \dfrac{1}{x}}+
\sqrt{y\cdot \dfrac{4}{y}}\right) ^2=3^2=9 $. 

\item 在平面四边形$ ABCD $中,已知$ \Delta ABC $的面积是$ \Delta ADC $
的面积的3倍,若存在正实数$ x,y $使得$ \vec{AC}=\left(\dfrac{1}{x}-3
\right)\vec{AB}+\left(1-\dfrac{1}{y} 
\right)\vec{AD} $成立,求$ x+y $的最小值。\\
\textbf{解}\ 对于$ \Delta ABC $和$ \Delta ADC $,如果都以对角线$ AC $为底边,
那么两个三角形的高的比值是3,再根据相似三角形,
$ BD $被$ AC $分成3:1两段。$ 3\left(\dfrac{1}{x}-3 \right)=\left(1-
\dfrac{1}{y} \right) $,$ \dfrac{3}{x}+\dfrac{1}{y}=10 $,
$ x+y=(x+y)\cdot\dfrac{1}{10}\Big(\dfrac{3}{x}+\dfrac{1}{y}\Big)=
\dfrac{1}{10}\Big(3+1+\dfrac{3y}{x}+\dfrac{x}{y}\Big)
\geq\dfrac{2+\sqrt{3}}{5} $. 

\item 设向量$ \vec{a},\vec{b},\vec{c} $
满足$ |\vec{a}|=|\vec{b}|=1,\vec{a}\cdot
\vec{b}=-\dfrac{1}{2},\langle\vec{a}-\vec{c},
\vec{b}-\vec{c}\rangle=\dfrac{\pi}{3} $,
则$ |\vec{c}| $的最大值是多少?
\begin{figure}[h]
    \centering
    \includegraphics[width=0.4\linewidth]{向量-四点共圆题}
\end{figure} \\
\textbf{解}\ $ \vec{O_1A}=\vec{a},\ \vec{O_1B}=
\vec{b},\angle AOB=120^\circ $,$ \vec{c} $的终点可能在$ \odot 
O_1 $或$ \odot O_2 $上,$ |\vec{c}| $最大是$ |O_1C_2|=2 $ .

\item 在$ \Delta ABC $中,$ \vec{AB}=(\sqrt{3}\cos x,\cos x),
\vec{AC}=(\cos x,\sin x) $,则$ \Delta ABC $面积的最大值是多少?\\
\textbf{解}\ 根据(\ref{三角形面积公式x1y2-x2y1})式,$ S_{\Delta ABC}=\dfrac{1}{2}
\left|\sqrt{3}\cos x\sin x-\cos^2 x\right|=\dfrac{1}{2}\left|\dfrac{\sqrt{3}}{2}
\sin 2x-\dfrac{1}{2}\cos 2x-\dfrac{1}{2}\right|
=\dfrac{1}{2}\left|\sin\Big(2x-\dfrac{\pi}{6}\Big)-\dfrac{1}{2}\right|
\leq \dfrac{3}{4} $. 

\item 已知点$ M,N $在以$ AB $为直径的圆上,若$ AB=5,AM=3,BN=2 $ ,
求$ \vec{AB}\cdot\vec{MN} $. 
\begin{figure}[H]
    \centering
    \includegraphics[width=0.3\linewidth]{向量题圆AB=5}
\end{figure}
\noindent\textbf{方法一}\ $ \vec{AB}\cdot\vec{MN}=\vec{AB}
\cdot(\vec{AN}-\vec{AM})=\vec{AB}\cdot
\vec{AN}-\vec{AB}\cdot\vec{AM}=
|AN|^2-|AM|^2=(|AB|^2-|BN|^2)-|AM|^2=25-4-9=12 $. \\
\textbf{方法二}\ $ \cos\angle AOM=\dfrac{7}{25},\ \cos\angle BON=\dfrac{17}{25} $,
以$ O $为原点,$ OB $为$ x $轴正半轴建立坐标系,则
$ M\left(-\dfrac{5}{2}\cdot\dfrac{7}{25},y_M\right) $,
$ N\left(\dfrac{5}{2}\cdot\dfrac{17}{25},y_N\right) $,
$ \vec{MN}=\left(\dfrac{12}{5},y_N-y_M\right) $,$ \vec{AB}=(5,0) $,
$ \vec{AB}\cdot\vec{MN}=5\cdot\dfrac{12}{5}=12 $. \\
\textbf{注}\ 不论$ M,N $两点分布在直径$ AB $同侧还是两侧,最终结果是一样的。

\item (2022,新高考I卷)已知$ \Delta ABC $满足$ \dfrac{\cos A}{1+\sin A}=
\dfrac{\sin 2B}{1+\cos 2B} $. \\
(1)若$ C=\dfrac{2\pi}{3} $,求$ B $;\\
(2)求$ \dfrac{a^{2}+b^{2}}{c^{2}} $的最小值。\\
\textbf{解}\ (1) 
\begin{gather*}
    \dfrac{\cos A}{1+\sin A}=\dfrac{\sin 2B}{1+\cos 2B}=
    \dfrac{2\sin B\cos B}{2\cos^2 B}=\dfrac{\sin B}{\cos B} \\
    \cos(A+B)=-\cos C=\sin B 
\end{gather*}
所以,$ C=B+\dfrac{\pi}{2} $,$ B=\dfrac{\pi}{6} $. \\
(2) $ A=\pi-B-C=\dfrac{\pi}{2}-2B>0 $,$ 0<B<\dfrac{\pi}{4} $,
\begin{align*}
    \dfrac{a^{2}+b^{2}}{c^{2}} &=\dfrac{c^2+2ab\cos C}{c^2}
    =1+\dfrac{2ab}{c^2}\cos C=1+\dfrac{2\sin A\sin B\cos C}{\sin^2 C} \\
    &=1+\dfrac{2\sin(\frac{\pi}{2}-2B)\sin B(-\sin B)}{\cos^2 B}
    =1+\dfrac{2\cos2B(-\sin^2 B)}{\cos^2 B} \\
    &=1+\dfrac{2(2\cos^2 B-1)(\cos^2B-1)}{\cos^2 B}
    =1+2\left(2\cos^2 B+\dfrac{1}{\cos^2B}-3\right) \\
    &\geq 1+2(2\sqrt{2}-3)=4\sqrt{2}-5
\end{align*}
当$ 2\cos^2 B=\dfrac{1}{\cos^2B},\ \cos B=\dfrac{1}{\sqrt[4]{2}}\in 
\left(\dfrac{\sqrt{2}}{2},1\right) $时,等号成立。

所以,$ \dfrac{a^{2}+b^{2}}{c^{2}} $的最小值是$ 4\sqrt{2}-5 $. 
%\\ \textbf{注}\ 本题的难点在于求$ \dfrac{2\cos2B(-\sin^2 B)}{\cos^2 B} $
%的最值,如果尝试求导,则会消耗过多时间,而且容易出错,在考场上不划算。
%本题在解三角形的题目中属于较难的,但我认为不能算优质的题目,
%因为求导才是求函数最值的通法,故意制造手工求导计算的困难,
%而迫使学生想其它方法,这不是正确的导向,特别是在手机和计算机
%十分普及的时代。现在的学生根本不应该在这些适用范围狭小的低端
%变形技巧上浪费时间,只要把$ \dfrac{2\cos2B(-\sin^2 B)}{\cos^2 B} $
%稍加改动,比如改成$ \dfrac{2\cos2.2B(-\sin^2 B)}{\cos^2 B},\ 
%\dfrac{2\cos2B(-\sin B)}{\cos^2 B} $,那么本题中的低端变形技巧
%就失效了,还是要靠求导,并用计算机求导函数零点。
%所以命题者和教师应该引导学生花时间学习更高级的、适用范围更广的通法。
%毕竟,人的精力是有限的,通法才是数学的正统。

\item (2022,全国乙卷)记$\triangle ABC$的内角$A, B, C$的对边分别为$a, b, c$,\\
已知 $\sin C\sin(A-B)=\sin B\sin(C-A)$. \\
(I) 证明:$2a^{2}=b^{2}+c^{2}$;\\
(II) 若$a=5,\ \cos A=\dfrac{25}{31}$,求$\triangle ABC$的周长。\\
\textbf{解}\ (I)
\begin{gather*}
    \sin C\sin(A-B)=\sin B\sin(C-A) \\
    \sin C\sin A\cos B-\sin C\cos A\sin B=
    \sin B\sin C\cos A-\sin B\cos C\sin A \\
    \sin A(\sin C\cos B+\cos C\sin B)=\sin^2A=2\sin B\sin C\cos A \\
    \cos A=\dfrac{\sin^2 A}{2\sin B\sin C}=\dfrac{a^2}{2bc}=
    \dfrac{b^2+c^2-a^2}{2bc} \\
    2a^{2}=b^{2}+c^{2}
\end{gather*}
(II) $ \cos A=\dfrac{a^2}{2bc}=\dfrac{25}{2bc}=\dfrac{25}{31} $,
$ 2bc=31 $,$ b^{2}+c^{2}=2a^2=50 $,$ (b+c)^2=50+31=81 $,$ b+c=9 $,
周长为$ a+b+c=5+9=14 $.

\item 四个不同的点$ A,B,C,D $都位于一条直线上,$ P $点不在这条直线上,求证:
\begin{gather*}
    \dfrac{|AC|\cdot|BD|}{|AD|\cdot|BC|}=
    \dfrac{\sin\angle APC\cdot \sin\angle BPD}
    {\sin\angle APD\cdot \sin\angle BPC}
\end{gather*}
\begin{figure}[!ht]
    \centering
    \includegraphics[width=0.3\linewidth]{线段交比等于角度正弦交比}
\end{figure}
\\
\textbf{证}\ 由正弦定理,
\begin{align}
    & \dfrac{|AC|}{\sin\angle APC}=\dfrac{|PA|}{\sin\angle PCA}
    =\dfrac{|PC|}{\sin\angle PAC} \label{线束交比正弦1} \\
    & \dfrac{|BD|}{\sin\angle BPD}=\dfrac{|PB|}{\sin\angle PDB} 
    =\dfrac{|PD|}{\sin\angle PBD} \label{线束交比正弦2}\\
    & \dfrac{|AD|}{\sin\angle APD}=\dfrac{|PD|}{\sin\angle PAC}
    =\dfrac{|PA|}{\sin\angle PDB} \label{线束交比正弦3}\\    
    & \dfrac{|BC|}{\sin\angle BPC}=\dfrac{|PC|}{\sin\angle PBC}
    =\dfrac{|PB|}{\sin\angle PCB} \label{线束交比正弦4}
\end{align}
设$ h $代表$ P $到$ A,B,C,D $所在直线的距离,所以,
\begin{align*}
    \dfrac{\dfrac{|AC|}{\sin\angle APC}\cdot \dfrac{|BD|}{\sin\angle BPD}}
    {\dfrac{|AD|}{\sin\angle APD}\cdot \dfrac{|BC|}{\sin\angle BPC}} &=
    \dfrac{\dfrac{|PA|}{\sin\angle PCA}\cdot \dfrac{|PB|}{\sin\angle PDB}}
    {\dfrac{|PD|}{\sin\angle PAC}\cdot \dfrac{|PC|}{\sin\angle PBC}} \\
    &=\dfrac{|PA|\sin\angle PAC\cdot |PB|\sin\angle PBC}
    {|PD|\sin\angle PDB\cdot |PC|\sin\angle PCA}=\dfrac{h\cdot h}{h\cdot h}=1 
\end{align*}
上面的计算使用了(\ref{线束交比正弦1})$ \sim $(\ref{线束交比正弦4})式的中间一列,
当然也可以使用最右边的列,请读者自行完成。

\item 设平面$ \alpha $为复平面,$ O $为坐标原点,以$ O $为球心作一个
半径为1的球面。复平面$ \alpha $上有两点$ A,B $,这两点对应的复数
分别为$ z_1,z_2 $. 线段$ ON\perp\alpha $且$ N $点位于球面上,
$ N,A $所在直线交球面于$ A' $点,$ N,B $所在直线交球面于$ B' $点。
求线段$ |A'B'| $的长度(用含$ z_1,z_2 $的式子表示。) 
\begin{figure}[!ht]
    \centering
    \includegraphics[width=0.5\linewidth]{黎曼球面度量}
\end{figure}
\\
\textbf{解}\ 因为$ |ON|=|OA'|=|OB'|=1 $,所以$ \Delta ONA',
\Delta ONB' $均为等腰三角形,且
\begin{align*}
    |NA'| =2\cos\angle ONA=\dfrac{2}{|NA|},\qquad
    |NB'| =2\cos\angle ONB=\dfrac{2}{|NB|} 
\end{align*}
在$ \Delta NAB $和$ \Delta NA'B' $中使用余弦定理,有
\begin{align*}
    |A'B'|^2 &=|NA'|^2+|NB'|^2-2|NA'|\cdot|NB'|\cos\angle ANB \\
    &=|NA'|^2+|NB'|^2-2|NA'|\cdot|NB'|
    \dfrac{|NA|^2+|NB|^2-|AB|^2}{2|NA|\cdot|NB|} \\
    &=\dfrac{4}{|NA|^2}+\dfrac{4}{|NB|^2}-
    \dfrac{4(|NA|^2+|NB|^2-|AB|^2)}{|NA|^2\cdot|NB|^2} \\
    &=\dfrac{4|AB|^2}{|NA|^2\cdot|NB|^2} 
\end{align*}
所以
\begin{gather*}
    |A'B'|=\dfrac{2|AB|}{|NA|\cdot|NB|}=
    \dfrac{2|z_1-z_2|}{\sqrt{1+|z_1|^2}\sqrt{1+|z_2|^2}}
\end{gather*}
\textbf{注}\ 事实上,
\begin{gather*}
    \dfrac{|NA'|}{|NB'|}=\dfrac{\dfrac{2}{|NA|}}{\dfrac{2}{|NB|}}
    =\dfrac{|NB|}{|NA|}
\end{gather*} 
所以,$ \Delta NB'A' \backsim \Delta NAB $,也可以利用相似三角形求$ |A'B'| $.\\
\textbf{变体}\ 如果不是球心在复平面上,而是球面与复平面相切于原点$ O $,
$ A',B',ON $定义同上,那么$ |A'B'| $表达式会怎样变化?\footnote{
    \q $ |A'B'|=\dfrac{2|AB|}{|NA|\cdot|NB|}=
    \dfrac{2|z_1-z_2|}{\sqrt{4+|z_1|^2}\sqrt{4+|z_2|^2}} $.  } 

\item $ ^* $
在$ \Delta ABC $中,$ D,E,F $分别是$ BC,AC,AB $上的三等分点,
且$ D $靠近$ C $点,$ E $靠近$ A $点,$ F $靠近$ B $点,
求$ \Delta PQR $与$ \Delta ABC $面积的比值。
\begin{figure}[!ht]
    \centering
    \includegraphics[width=0.7\linewidth]{"1_7面积占比 向量题"}
\end{figure}
\\
\textbf{方法一}\ 向量法:\\
\begin{gather*}
    \vec{AP}=\lambda \left(\dfrac{1}{3}\vec{AC} 
    \right) +(1-\lambda)
    \vec{AB}=k_1\vec{AD}=k_1\left[
    \dfrac{2}{3}\vec{AC}+
    \dfrac{1}{3}\vec{AB}\right] \\
    \dfrac{\dfrac{1}{3}\lambda}{1-\lambda}=\dfrac{2}{1},
    \lambda=\dfrac{6}{7},k_1=\dfrac{3}{7} \\
    \vec{AR}=\mu \left(\dfrac{2}{3}\vec{AB} \right) +(1-\mu)\vec{AC}=k_2\vec{AD}=k_2\left[ \dfrac{2}{3}\vec{AC}+
    \dfrac{1}{3}\vec{AB} \right] \\
    \dfrac{\dfrac{2}{3}\mu}{1-\mu}=\dfrac{1}{2},\mu=\dfrac{3}{7},
    k_2=\dfrac{6}{7} 
\end{gather*}
所以,$ P $是$ AR $的中点,同理可得,$ Q $是$ BP $的中点,$ R $是$ CQ $的中点,于是
\begin{gather*}
    \dfrac{S_{\Delta PQR}}{S_{\Delta ABC}} =\dfrac{|BD|}{|BC|}\cdot
    \dfrac{|PR|}{|AD|}\cdot \dfrac{|PQ|}{|PB|}=\dfrac{2}{3}\cdot \dfrac{3}{7}\cdot
    \dfrac{1}{2} =\dfrac{1}{7}
\end{gather*}
\\
\textbf{方法二}\ 相似三角形法:过$ D $作$ CF $的平行线交$ AB $于$ J $点,过$ R,D $作$ BE $的平行线,
交$ AC $于$ K,L $两点,
\begin{gather*}
    |CL|:|LE|=|CD|:|DB|=1:2,\ |EL|=\dfrac{2}{3}|EC|=
    \dfrac{2}{3}\left(\dfrac{2}{3} |AC|\right) =\dfrac{4}{9}|AC|,\\
    |AP|:|PD|=|AE|:|EL|=3:4 \\
    |BJ|:|JF|=|BD|:|DC|=2:1,\ |JF|=\dfrac{1}{3}|BF|=
    \dfrac{1}{3}\left(\dfrac{1}{3}|AB| \right)=\dfrac{1}{9}|AB|, \\
    |AF|:|FJ|=|AR|:|RD|=|AK|:|KL|=6:1  
\end{gather*}
所以,$ |EK|=\dfrac{1}{3}|AC| $,$ K $是$ EC $中点,
$ P $是$ AR $的中点,剩余步骤同方法一。\\
\textbf{注1}\ 如果把三等分点改成$ n $等分点,那么$     \mu=k_1=\dfrac{n}{n^2-n+1},\ \lambda=k_2=\dfrac{n^2-n}{n^2-n+1} $,
\begin{gather*}
    \dfrac{S_{\Delta PQR}}{S_ {\Delta ABC}} =\dfrac{|BD|}{|BC|}\cdot
    \dfrac{|PR|}{|AD|} \cdot \dfrac{|PQ|}{|PB|}=\dfrac{n-1}{n}
    \cdot(k_2-k_1)\cdot\dfrac{k_2-k_1}{k_2}=\dfrac{(n-2)^2}{n^2-n+1}
\end{gather*}
如果$ n=2 $,那么$ S_{\Delta PQR} =0 $,这正是三条中线交于一点的表现。\\
\textbf{注2}\ 更一般地,若$ \dfrac{|AF|}{|FB|}=\lambda_1,\ 
\dfrac{|BD|}{|DC|}=\lambda_2,\ \dfrac{|CE|}{|EA|}=\lambda_3 $,那么
\begin{align*}
    \dfrac{S_{\Delta PQR}}{S_ {\Delta ABC}} =\dfrac{(\lambda_1\lambda_2
        \lambda_3-1)^2}{(1+\lambda_1+\lambda_1\lambda_2)
        (1+\lambda_2+\lambda_2\lambda_3)(1+\lambda_3+\lambda_3\lambda_1)}
\end{align*}

\item 已知$ \vec{a}=(x,t),\ \vec{b}=(t,y) $,
若$ x,y $是常数,且$ \vec{a},\vec{b} $的夹角为$ \theta $,
求$ t $. \\
\textbf{方法一}\ 设向量$ \vec{a},\vec{b} $所在的直线倾斜角
分别为$ \theta_1,\theta_2 $,由正切差角公式:
\begin{gather*}
    \pm\tan\theta=\tan(\theta_1-\theta_2)=\dfrac{\tan\theta_1-\tan\theta_2}
    {1+\tan\theta_1	\tan\theta_2}=\dfrac{\dfrac{t}{x}-\dfrac{y}{t}}{1+
        \dfrac{t}{x}\cdot\dfrac{y}{t}}=\dfrac{t^2-xy}{xt+yt} 
\end{gather*}
\begin{gather}\label{正切化成的二次方程}
    t^2\pm(\tan\theta)(x+y)t-xy=0
\end{gather}
直接解方程,就能得到$ t $. 
现在让我们站在命题人的角度来看方程(\ref{正切化成的二次方程}),我们希望解出的$ t $
是有理数(不含有根式),最好还是整数,以减小计算量。为满足这个要求,该如何选定$ x,y $呢?
让我们从韦达定理的角度考虑。\\
\mycircled{1} 方程(\ref{正切化成的二次方程})的两根为$ x,-y $,那么
\begin{gather}\label{考虑有理数-情形1}
    x-y=\pm(\tan\theta)(x+y)
\end{gather}
只要让$ \tan\theta $是有理数,那么当$ x $是有理数时,$ y $也必然是有理数。\\
\mycircled{2} 方程(\ref{正切化成的二次方程})的两根为$ 1,-xy $,那么
\begin{gather}\label{考虑有理数-情形2}
    1-xy=\pm(\tan\theta)(x+y)
\end{gather}
只要让$ \tan\theta $是有理数,那么当$ x $是有理数时,$ y $也必然是有理数。\\
\mycircled{3} 方程(\ref{正切化成的二次方程})的两根为$ \sqrt{x},-\sqrt{x}y $,那么
\begin{gather}\label{考虑有理数-情形3}
    \sqrt{x}(1-y)=\pm(\tan\theta)(x+y) 
\end{gather}
只要让$ \tan\theta $是有理数,那么当$ \sqrt{x} $是有理数时,$ y $也必然是有理数。\\
\\
\textbf{方法二}\ $ \cos\theta=\dfrac{tx+ty}{\sqrt{x^2+t^2}\sqrt{y^2+t^2}} $,整理得:
\begin{align}\label{余弦化成t2二次方程}
    t^4+\left[x^2+y^2-\dfrac{1}{\cos^2\theta}(x+y)^2\right]t^2+x^2y^2=0
\end{align}
这是关于$ t^2 $的二次方程。\\
\mycircled{1} 方程(\ref{余弦化成t2二次方程})的两根为$ x^2,y^2 $,那么
\begin{gather*}
    x^2+y^2=\dfrac{1}{\cos^2\theta}(x+y)^2-(x^2+y^2) \\
    2(x^2+y^2)-(x+y)^2=\dfrac{1}{\cos^2\theta}(x+y)^2 -(x+y)^2 \\
    (x-y)^2=(\tan\theta)^2(x+y)^2
\end{gather*}
这与(\ref{考虑有理数-情形1})式等价。\\
\mycircled{2} 方程(\ref{余弦化成t2二次方程})的两根为$ 1,x^2y^2 $,那么
\begin{gather*}
    1+x^2y^2=\dfrac{1}{\cos^2\theta}(x+y)^2-(x^2+y^2) \\
    1+x^2y^2-2xy=\dfrac{1}{\cos^2\theta}(x+y)^2-(x^2+y^2+2xy) \\
    (1-xy)^2=(\tan\theta)^2 (x+y)^2 
\end{gather*}
这与(\ref{考虑有理数-情形2})式等价。\\
\mycircled{3} 方程(\ref{余弦化成t2二次方程})的两根为$ x,xy^2 $,那么
\begin{gather*}
    x+xy^2=\dfrac{1}{\cos^2\theta}(x+y)^2-(x^2+y^2) \\
    x+xy^2-2xy=\dfrac{1}{\cos^2\theta}(x+y)^2-(x^2+y^2+2xy) 
\end{gather*}
\begin{gather}\label{考虑有理数-情形3已平方}
    x(1-y)^2=(\tan\theta)^2(x+y)^2
\end{gather}
这与(\ref{考虑有理数-情形3})式等价。列举一些整数解的情形:$ \tan\theta=3 $,
(\ref{考虑有理数-情形3已平方})式有$ (1,-2),(4,-14) $等整数解;
$ \tan\theta=4 $,(\ref{考虑有理数-情形3已平方})式有$ (4,-9),(9,-39) $等整数解;
$ \tan\theta=5 $,(\ref{考虑有理数-情形3已平方})式有
$ (9,-24),(16,-84) $等整数解。

\item 正实数$ x,y,z $满足$ \left\{ \begin{aligned} x^2+xy+y^2 =&\ 9  &\mycircled{1} \\
    y^2+yz+z^2=&\ 16 &\mycircled{2} \\
    z^2+zx+x^2=&\ 25 &\mycircled{3} \end{aligned} \right. $,求$ xy+yz+zx $的值。\\
\textbf{解}\ $ x,y,z $可看成边长为$ 3,4,5 $的直角三角形的费马点与三个顶点连线的长度,
直角三角形的面积等于三个小三角形面积之和:$ \dfrac{1}{2}\cdot 3\cdot 4= \dfrac{1}{2}\cdot(xy+yz+zx)\sin 120^{\circ} $,于是
$ xy+yz+zx=8\sqrt{3} $. 如果将\mycircled{1},\mycircled{2},\mycircled{3}
式等号右侧的常数分别换成$ a^2,b^2,c^2 $,那么利用海伦公式可得$ xy+yz+zx=
\dfrac{4}{\sqrt{3}}\sqrt{p(p-a)(p-b)(p-c)} $,其中$ p=\dfrac{a+b+c}{2} $. 

如果读者第一次遇到此类较难的题目而无从下手,但题目恰好以选择题形式出现,
那么至少可利用基本不等式排除明显错误的答案,因为$ x,y,z $任意两个都不相等
(令其中两个相等,会得到相互矛盾的两个方程),
所以,$ 2xy+xy < x^2+y^2+xy =9,\ xy <3$,同理可得,$ yz<\dfrac{16}{3},zx<\dfrac{25}{3} $,
于是$ xy+yz+zx < \dfrac{50}{3} $,所有大于等于$ \dfrac{50}{3}=16.66\cdots $
的选项均可排除。$ (8\sqrt{3}\approx 13.8564 ) $. 

如果将$ 9,16,25 $换成一般性的常数$ a^2,b^2,c^2\ 
(a+b>c>0,b+c>a>0,c+a>b>0) $,那么有
\begin{gather*}
    xy+yz+zx\leq \dfrac{1}{3}(a^2+b^2+c^2)
\end{gather*}
等号在$ a=b=c $时成立。以上不等式还可以从另一个角度理解,
结合$ xy+yz+zx=\dfrac{4}{\sqrt{3}}S_{\Delta ABC} $与外森比克不等式
(\ref{外森比克不等式}),
\begin{gather*}
    S_{\Delta ABC}\leq \dfrac{1}{4\sqrt{3}}(a^2+b^2+c^2) \\
    xy+yz+zx=\dfrac{4}{\sqrt{3}}S_{\Delta ABC}\leq 
    \dfrac{4}{\sqrt{3}}\cdot \dfrac{1}{4\sqrt{3}}(a^2+b^2+c^2)=
    \dfrac{1}{3}(a^2+b^2+c^2)
\end{gather*}

如果想在只有计算器的情况下求这个方程组的近似解,
可用消元法转化为关于一个变量的方程,然后用二分法求解。
例如,由\mycircled{1}可得$ x=\dfrac{1}{2}(-y\pm\sqrt{36-3y^2}) $,由\mycircled{2}可得
$ z=\dfrac{1}{2}(-y\pm\sqrt{64-3y^2}) $,将$ x,z $的表达式(只考虑取正根的情况)带入\mycircled{3},整理后可得:
\begin{align}\label{费马点例题f(y)=3y2+3y}
    f(y)=3y^2+3y(\sqrt{36-3y^2}+\sqrt{64-3y^2})-\sqrt{36-3y^2} \sqrt{64-3y^2}=0
\end{align}
此方程已经可以通过二分法求解(也可用GeoGebra求零点)。当然,还可以多次进行平方与移项操作,
消去所有的根号,得到一个关于$ y $的高次方程。解出$ y $以后就立刻能得到$ x,z $. 
但如果只是要得到$ xy+yz+zx $的近似值,是可以不直接使用$ x,z $的值的,这是因为
\begin{align*}
     &\ xy+yz+zx=(x+z)y+zx\\ =&\ \dfrac{1}{4}
     \left[-3y^2+y(\sqrt{36-3y^2}+\sqrt{64-3y^2})  
    +\sqrt{36-3y^2}\cdot \sqrt{64-3y^2}\right]\ 
     (\text{利用}(\ref{费马点例题f(y)=3y2+3y})\text{化简})\\ 
    =&\ y(\sqrt{36-3y^2}+\sqrt{64-3y^2})
\end{align*}

通过求导,可算出 $ f(y) $的极小值点为
$ \left( -\dfrac{12}{\sqrt{37}},-96\right) $,极大值点为
$ \left(\dfrac{12}{\sqrt{13}},96\right) $,$ f(y) $的图像如下,形状像一把勺子,
并不具有任何对称性,但极大值和极小值的绝对值相同,还是令人有些诧异。
感兴趣的读者还可尝试利用初等方法(不借助导数)寻找$ f(y) $的极值。 
\begin{figure}[h]
    \centering
    \includegraphics[width=0.6\linewidth]{xy+yz+zx题目}
\end{figure}

下面介绍由\mycircled{1}、\mycircled{3}、\mycircled{3}
组成的方程组的直接求解方法\footnote{作者的好友Kimmo提供了一些帮助。},
将$ 9,16,25 $换成一般性的常数$ a^2,b^2,c^2 $,(不要求$ a^2+b^2=c^2 $,
只要求以$ |a|,|b|,|c| $为边长能构成三角形。)
分别用$ \mycircled{1}-\mycircled{2},
\mycircled{1}-\mycircled{3},\mycircled{2}-\mycircled{3} $,
(这一方法已经在第\pageref{复数多元二次方程求解演示} 页的例题的方法三中演示过了,
形式一致的式子相减,往往可以因式分解),可得:
\begin{align*}
    \left\{
    \begin{aligned}
        (x+y+z)(x-z)=&\ a^2-b^2   \\
        (x+y+z)(y-z)=&\ a^2-c^2  \\
        (x+y+z)(y-x)=&\ b^2-c^2   
    \end{aligned}
    \right.
\end{align*}
以上三式左右同时平方后相加,可得:
\begin{gather*}
    \left[ x^2+y^2+z^2+2(xy+yz+zx) \right] \left[ 2(x^2+y^2+z^2)-2(xy+yz+zx) \right] \\
    =2(a^4+b^4+c^4)-2(a^2b^2+b^2c^2+c^2a^2) \hspace{1cm} \mycircled{4}
\end{gather*}
由 \mycircled{1}+\mycircled{2}+\mycircled{3} 可得
\begin{align*}
    2(x^2+y^2+z^2)+(xy+yz+zx)=a^2+b^2+c^2  \hspace{1cm} \mycircled{5}
\end{align*}
记
\begin{align*}
    & s_1=x+y+z \\
    & s_2=x^2+y^2+z^2 \\
    & \sigma_2=xy+yz+zx  \\
    & u=a^4+b^4+c^4 \\
    & v=a^2b^2+b^2c^2+c^2a^2 \\
    & w=a^2+b^2+c^2
\end{align*}
那么$ w^2=u+2v $,\mycircled{4}、\mycircled{5}两式成为
\begin{align*}
    \left\{
    \begin{aligned}
        (s_2+2\sigma_2)(2s_2-2\sigma_2)=&\ 2u-2v \\
        2s_2+\sigma_2 =&\ w
    \end{aligned}
    \right.
\end{align*}
解得:
\begin{align*}
    \left\{
    \begin{aligned}
        s_2 =&\  \dfrac{1}{2}\left[w \pm \sqrt{\dfrac{-u+2v}{3}}\right] \\
        \sigma_2=&\ \pm \sqrt{\dfrac{-u+2v}{3}}=
        \pm \dfrac{4}{\sqrt{3}}\sqrt{\dfrac{-u+2v}{16}}=
        \pm\dfrac{4}{\sqrt{3}}S_{\Delta ABC} 
        %    =\pm \dfrac{4}{\sqrt{3}}\sqrt{p(p-a)(p-b)(p-c)}
    \end{aligned}
    \right.
\end{align*}
所以,
\begin{gather*}
    s_1=x+y+z=\pm\sqrt{s_2+2\sigma_2 }=\pm \sqrt{\dfrac{1}{2}[w\pm\sqrt{3(-u+2v)}]}
\end{gather*}
因为$ x=\dfrac{1}{2}(-y\pm\sqrt{4a^2-3y^2}) , z=\dfrac{1}{2}(-y\pm\sqrt{4b^2-3y^2}) $,
根式前的正负号选取有多种情况,先考虑其中一种,有
\begin{align*}
    x+y+z=\dfrac{1}{2}\left(\sqrt{4a^2-3y^2}+\sqrt{4b^2-3y^2} \right)=s_1
\end{align*}
可解得
\begin{align*}
    y=&\  \pm \sqrt{\dfrac{1}{3}\left[2(a^2+b^2)-s_1^2-\dfrac{(b^2-a^2)^2}{s_1^2} \right] }\\
    =&\  \pm \sqrt{\dfrac{1}{3}\left\{2(a^2+b^2)-\dfrac{1}{2}\left[w\pm\sqrt{3(-u+2v)}\right] -\dfrac{2(b^2-a^2)^2}{w\pm\sqrt{3(-u+2v)}} \right\} } 
\end{align*}
当$ a,b,c $分别为$ 3,\ 4,\ 5 $时,若限制$ x,y,z $均为正数,则
\begin{align*}
    \left\{
    \begin{aligned}
        x =&\  -2\sqrt{\dfrac{3(25-12\sqrt{3})}{193}}+
        3\sqrt{\dfrac{3(31+16\sqrt{3})}{193}}=2.35400309 \cdots \\
        y =&\ 4\sqrt{\dfrac{3(25-12\sqrt{3})}{193}}= 1.02390782 \cdots \\
        z =&\ -2\sqrt{\dfrac{3(25-12\sqrt{3})}{193}}+
        2\sqrt{\dfrac{547+108\sqrt{3}}{193}}= 3.38852164 \cdots 
    \end{aligned}
    \right.
    \Rightarrow 
    \left\{
    \begin{aligned}
        s_2 =&\ 25- 4\sqrt{3} \\
        \sigma_2 =&\ 8\sqrt{3} \\
        s_1=&\ \sqrt{25 + 12\sqrt{3}}
    \end{aligned}
    \right.
\end{align*}
若允许$ x,y,z $出现负数,则还有如下一组解:
\begin{align*}
    \left\{
    \begin{aligned}
        x =&\  2\sqrt{\dfrac{3(25+12\sqrt{3})}{193}}-
        3\sqrt{\dfrac{3(31-16\sqrt{3})}{193}}=1.00908617 \cdots \\
        y =&\ -4\sqrt{\dfrac{3(25+12\sqrt{3})}{193}}=-3.37444009 \cdots \\
        z =&\ 2\sqrt{\dfrac{3(25+12\sqrt{3})}{193}}+
        2\sqrt{\dfrac{547-108\sqrt{3}}{193}}= 4.41849548 \cdots 
    \end{aligned}
    \right.
    \Rightarrow 
    \left\{
    \begin{aligned}
        s_2 =&\ 25+ 4\sqrt{3} \\
        \sigma_2 =&\ -8\sqrt{3} \\
        s_1=&\ \sqrt{25 - 12\sqrt{3}}
    \end{aligned}
    \right.
\end{align*}
当然,如果把$ x,y,z $全部变成相反数,也是原方程组的解。
\footnote{注:$ \sqrt{25\pm 12\sqrt{3}} =\dfrac{1}{2}\left(\sqrt{50+2\sqrt{193}}
    \pm \sqrt{50-2\sqrt{193}}\right) $,无法去掉外层根号。}

\begin{figure}[H]
    \centering
    \includegraphics[width=0.75\linewidth]{费马点3-4-5}
\end{figure} 
第一组解对应上面左图的情形:向$ \Delta ABC $外部作等边三角形,三个外接圆交点
$ P $在$ \Delta ABC $内部,$ \angle APB=\angle BPC =\angle CPA
=120^{\circ} ,S_{\Delta ABC}=S_{\Delta APB}+S_{\Delta BPC}+S_{\Delta CPA}$.
第二组解对应上面右图的情形:向$ \Delta ABC $内部作等边三角形,
三个外接圆交点$ P $在$ \Delta ABC $外部,$ \angle APB=\angle APC
=60^{\circ},\angle BPC=120^{\circ},S_{\Delta ABC}=S_{\Delta APB}+
S_{\Delta APC}-S_{\Delta BPC} $. 在两张图中都有:
$ |AC|=3,|AB|=4,|BC|=5, |PC|=|x|,|PA|=|y|,|PB|=|z| $. 


方程\mycircled{1}、\mycircled{2}、\mycircled{3}还可以看成三维空间中
三个两两相互垂直的椭圆柱面,感兴趣的可以用GeoGebra画图看看。
关于$ x+y+z $还有如下不等式\footnote{证明过程可查阅:张善立.
    有关费尔马点的一个不等式的加强[J]. 中等数学, 1997(04):22-22.}:
$ x+y+z\leq \sqrt{ab+bc+ca} $. 

\item $ ^* $ \label{2010江西高考}
(2010,江西高考)证明以下命题:\\
(I)对任意正整数$ a $ ,都存在正整数$ b,c\ (b<c) $,使得$ a^2,b^2,c^2 $为等差数列;\\
(II)存在无穷多互不相似的三角形$ \Delta_n $,其边长$ a_n,b_n,c_n $为正整数且
$ a_n^2,b_n^2,c_n^2 $成等差数列。\\
\textbf{解}\ (I)取$ b=5a,\ c=7a $,那么$ a^2+c^2=a^2+(7a)^2=50a^2=
2(5a)^2=2b^2 $. \\
(II) $ a_n^2+c_n^2=2b_n^2 $,与勾股定理形式相似,考虑从恒等式
\begin{gather*}
    (x^2-y^2)^2+(2xy)^2=(x^2+y^2)^2
\end{gather*}
出发,构造符合题意的正整数。上式两边同乘2,
\begin{gather*}
    2[(\underbrace{x^2-y^2}_{r})^2+(\underbrace{2xy}_{s})^2]=2(x^2+y^2)^2
\end{gather*}
因为$ 2(r^2+s^2)=(r-s)^2+(r+s)^2 $,所以上式可以变成
\begin{gather*}
    (x^2-y^2-2xy)^2+(x^2-y^2+2xy)^2=2(x^2+y^2)^2
\end{gather*}
取$ x_n,y_n\in \textbf{N}^+,\ x_n\geq 3y_n $,令$ a_n=x_n^2-y_n^2-2x_ny_n,\ 
b_n=x_n^2+y_n^2,\ c_n=x_n^2-y_n^2+2x_ny_n $,那么$ a_n^2,b_n^2,c_n^2 $
成等差数列,$ a_n=y_n^2\left(\dfrac{x_n^2}{y_n^2}-1-\dfrac{2x_n}{y_n}\right)
>0\ $ (当$ \dfrac{x_n}{y_n}\geq 3 $),下面需要证明这些三角形互不相似。
假设三角形$ \Delta_m,\Delta_n $相似,即$ \dfrac{a_m}{c_m}=
\dfrac{a_n}{c_n} $,那么
\begin{align*}
    \dfrac{x_m^2-y_m^2-2x_my_m}{x_m^2-y_m^2+2x_my_m} &=
    \dfrac{x_n^2-y_n^2-2x_ny_n}{x_n^2-y_n^2+2x_ny_n}  \\
    \dfrac{\dfrac{x_m^2}{y_m^2}-1-\dfrac{2x_m}{y_m}}
    {\dfrac{x_m^2}{y_m^2}-1+\dfrac{2x_m}{y_m}} &=
    \dfrac{\dfrac{x_n^2}{y_n^2}-1-\dfrac{2x_n}{y_n}}
    {\dfrac{x_n^2}{y_n^2}-1+\dfrac{2x_n}{y_n}} 
\end{align*}
因为函数$ \dfrac{u^2-1-2u}{u^2-1+2u}=1-\dfrac{4u}{u^2-1+2u}=1-
\dfrac{4}{u-\dfrac{1}{u}+2} $在$ u\in (1,+\infty) $时单调递增,
所以$ \dfrac{x_m}{y_m}=\dfrac{x_n}{y_n} $. 换言之,只要选取
$ \dfrac{x_m}{y_m}\neq\dfrac{x_n}{y_n} $,就能保证三角形
$ \Delta_m,\Delta_n $不相似,这样的选取方法显然是有无穷多的。\\
\textbf{注}\ 如果从恒等式$ (n^2-1)^2+(2n)^2=(n^2+1)^2 $出发,
推导过程基本一样。

\item 下图是往复式内燃机\footnote{以内燃机为动力的汽车与以电动机为动力
    的汽车将长期共存,内燃机不会退出历史的舞台,依然具有研究价值。
    160kWh的电量大致能让一般的5座电动私家车以100km/h的速度行驶800km,
    才能与燃油车的续航里程相抗衡,假设充电速度也要媲美加油速度,
    比如5分钟充满160kWh,那么充电功率高达$ 160\times 12=1920 $kW,
    这对电池、充电桩(线)、输电网都提出了巨大的挑战。假设一个家庭同时开2台空调,
    用电功率4kW,那么1920kW相当于480户家庭的用电功率。
    不同品牌的电动汽车的电池形状、接口、电芯类型、冷却方式很难统一,
    所以换电模式也是不同品牌各自为战,无法普及。}
%$ \sqrt{1.92}=1.385 $,这意味着充电电压超过$ 1385 $V,
%那么需要使用电阻率更高或者更厚的绝缘材料,否则会击穿。
%或者充电电流超过$ 1385 $A,那么需要使用电阻率更低或者是更粗的导线,
%否则发热量过大。
%}
的曲轴-连杆-活塞的简化图,$ O $是曲轴中心,
$ A $点代表曲柄销中心,$ A $点绕$ O $点旋转,$ |OA|=R $,
称为曲轴回转半径。$ AB $代表连杆,$ B $点代表活塞,只能沿直线运动,
$ |AB|=L>R $.设$ \angle AOB=\theta $, $ \angle ABO=\alpha $, 
$ |OB|=x $,求$ x $关于$ \theta $的关系式。
\begin{figure}[h]
\centering
\includegraphics[width=0.5\linewidth]{曲轴-连杆机构}
\end{figure} \\
\textbf{方法一}\ 过$ A $点作$ OB $的垂线,垂足为$ C $,则
$ R\sin\theta=L\sin\alpha $,
\begin{align*}
x =R\cos\theta+L\cos\alpha=R\cos\theta+\sqrt{L^2-L^2\sin^2\alpha} 
=R\cos\theta+\sqrt{L^2-R^2\sin^2\theta}
\end{align*}
\textbf{方法二}\ 由余弦定理,$ L^2=R^2+x^2-2Rx\cos\theta $,
求解二次方程,抛弃掉一个解,也可以得到与方法一相同的结果。

$ R\cos\theta+\sqrt{L^2-R^2\sin^2\theta} $与$ R\cos\theta+L $
的函数图像对比如下,$ \dfrac{R}{L} $越小,两者的图像越接近。
\begin{figure}[h]
\centering
\includegraphics[width=0.5\linewidth]{曲轴-连杆机构函数图像}
\end{figure} 
\\
\textbf{变体}\ 如果$ B $点运动所在的直线(气缸中心轴)不经过$ O $点,
而与$ O $点的距离为$ d $,求$ x=|DB| $关于$ \theta $的关系式。
\begin{figure}[h]
\centering
\includegraphics[width=0.5\linewidth]{曲轴-连杆机构-偏置}
\end{figure} \\
\textbf{解}\ $ R\sin\theta+d=L\sin\alpha $,
$ x=R\cos\theta+L\cos\alpha=R\cos\theta+\sqrt{L^2-(R\sin\theta+d)^2} $.

这种设计方案被称为“曲轴偏置”(分为正偏置与负偏置),正偏置的优点有
\footnote{$\diamond$ 顾丽,吴东兴,李波,等. 曲轴偏置和活塞销偏置对发动机摩擦损耗的影响研究[J]. 现代车用动力, 2018(3):3.\\
$\diamond$ 田丰果,睢娟. 曲轴偏置式发动机力学分析研究[J]. 数字技术与应用, 2009. \\
$\diamond$李永纯,张颖,袁海马.某型号发动机曲轴偏置分析.汽车工程师.(2014):44-46}:
在做功冲程中,连杆与气缸中心轴的夹角更小,
使发动机输出的扭矩和功率和热效率提高。同时可减小活塞环对气缸壁的正压力,从而减小摩擦力。

如果假设曲轴匀速旋转,即$ \theta=\omega t $,那么将$ x $对时间$ t $
求一阶导数和二阶导数,就能得到活塞往复运动的速度和加速度的变化规律。分析$ x $
的极大、极小值,就能得到活塞的行程,感兴趣的读者自行研究。

\item 设$ ABCD $是一个四边形,其中$ AD $边固定,另外三边可以运动
(但不会散开),设$ |AB|=l_1 $,$ |BC|=l_2 $,$ |CD|=l_3 $,
$ |DA|=l_4 $,假设已知$ \gamma $角,求$ \alpha,\ \beta $.
\begin{figure}[h]
\centering
\includegraphics[width=0.3\linewidth]{四连杆机构}
\end{figure} \\
\textbf{解}\ 
\begin{align*}
\begin{cases}
    l_1\sin\alpha +l_2\sin\beta=l_3\sin\gamma \\
    l_1\cos\alpha +l_2\cos\beta=l_4+l_3\cos\gamma 
\end{cases}
\end{align*}
以上两式平方后相加可得:
\begin{gather*}
l_1^2+l_2^2+2l_1l_2\cos(\alpha-\beta)=
l_3^2+l_4^2+2l_3l_4\cos\gamma =|AC|^2 \\
\alpha-\beta=\arccos\frac{l_3^2+l_4^2+2l_3l_4\cos\gamma-
    l_1^2-l_2^2}{2l_2l_2}
\end{gather*}
这样就相当于消去了一个变量,剩余的计算过程比较容易,在此略去。

本题中的可变形四边形是生活中的四连杆机构的抽象,
四连杆机构在很多机械设备中都有应用。
比如汽车转向系统中最常见的的阿克曼转向系统,
就是一个四连杆机构(其中有两根杆的长度相等),
作用是让汽车在转向时,弯道内侧车轮的偏转角大于外侧车轮,
尽量让内侧车轮和外侧车轮具有共同的转向中心,减少轮胎磨损。

\item 假设你偶遇了迪迦奥特曼,想测量他的身高。如下图所示,你位于左侧大楼
的$ A $点,大楼$ CD $对迪迦腿部有遮挡。你手上只有一台激光测距仪,
此仪器只能测量距离,无法测量角度。请设计一种测量方案,
测出迪迦当前站姿的身高$ EF $(不必考虑两腿没有并拢)。
\begin{figure}[h]
    \centering
    \includegraphics[width=0.6\linewidth]{迪迦身高计算(无测量方案)}
\end{figure} \\
\textbf{解}\ 测量方案如下:
\begin{figure}[h]
    \centering
    \includegraphics[width=0.6\linewidth]{迪迦身高计算(有辅助线)}
\end{figure} \\
测出$ |AE|=d_1 $, $ |AC|=d_2 $, $ |AH|=d_3 $(注意不是$ |CH|=d_3 $),
$ |AD|=d_4 $, $ |AB|=d_5 $,即可得到$ EF $.
但因为$ H $点是$ AC $延长线与$ EF $的交点,该点位于迪迦的体内,
激光测距仪无法测量。所以需要观测者沿垂直于纸面的方向,
平移测距仪的位置,以迪迦腹部表面的点来代替体内的点,进而测出$ |AH| $.
计算如下:过$ A $点作水平线,交$ CD $于$ G $点,交$ EF $于$ I $点,
则$ |AG|=|BD|=\sqrt{d_4^2-d_5^2} $.因为$ \Delta AGC \backsim \Delta AIH $,
所以
\begin{gather*}
    \dfrac{|AC|}{|AH|}=\dfrac{d_2}{d_3}=\dfrac{|AG|}{|AI|},\q  |AI|=\dfrac{d_3\sqrt{d_4^2-d_5^2} }{d_2},\q |EI|=\sqrt{d_1^2-|AI|^2}
\end{gather*}
最终有
\begin{gather*}
    |EF|=|EI|+|IF|=\sqrt{d_1^2-\frac{d_3^2(d_4^2-d_5^2)}{d_2^2}}+d_5
\end{gather*}
\textbf{注1}\ 以上方案也存在一些不足,如果大楼$ CD $太矮或者$ A $点位置太高,
那么$ H $点将位于迪迦的两腿之间,是一个悬空的点,从而无法测距。
当然,如果把迪迦换成一栋长方体形的大楼,那就不存在这个麻烦了。\\
\textbf{注2}\ 假如还可以测量角度,那么是否可以避免“要测量到悬空的点的距离”的困难
\footnote{可以避免。不测$ |AH| $,改为测量$ \angle EAD $即可,因为$ \angle 
    GAD=\arcsin\dfrac{d_5}{d_4} $是已知的,那么$ \angle EAG=\angle EAD-\angle 
    GAD $,$ |EI|=d_1\sin\angle EAG $. }?\\
\textbf{注3}\ 假如没有大楼$ CD $的遮挡(但不能测量角度),可以测量到$ F $点的距离,
又该如何测量身高\footnote{设$ |AF|=d_6 $,再利用$ d_1,d_5 $即可,
    $ |EF|=\sqrt{d_1^2-(d_6^2-d_5^2)}+d_5 $. }?

\end{enumerate}
%~\newpage

\section{习题}
\begin{enumerate}[leftmargin=\inteval{\myenumleftmargin}pt]
\item 在$ \Delta ABC $中,已知$ a^2+b^2+c^2=2\sqrt{3}bc\sin A $,求$ B $的大小。
\ifteach \\ \textbf{解}\ 
根据外森比克不等式,$ \Delta ABC $为等边三角形,$ B=\dfrac{\pi}{3} $. 
\fi

\item $ \vec{a}=(4,t),\ \vec{b}=(t,\dfrac{5}{3}) $,若
$ \vec{a}//\vec{b} $,则$ t= $\underline{\hspace{2cm}};
若$ \vec{a},\vec{b} $夹角为$ \dfrac{\pi}{4} $,
则$ t= $\underline{\hspace{2cm}}.
\ifteach \\ \textbf{解}\ 
$ \pm\dfrac{2\sqrt{15}}{3} $,1或$ \dfrac{20}{3} $,不能取负数,负数对应$ \dfrac{3\pi}{4} $. 
\fi

\item 平面直角坐标系中有4点,它们的坐标分别为$ A(3,4),B(6,0),C(7,2),D(4,6) $,
求四边形$ ABCD $的面积。
\ifteach \\
\textbf{解}\ $ \vec{AB}=\vec{DC}=(3,-4),
\vec{AD}=\vec{BC}=(1,2),\ S=|-4-6|=10 $. 
\fi

\item 在$ \Delta ABC $中,$ \sin^2A\leq\sin^2B+\sin^2C-\sin B\sin C $,
求$ A $的取值范围。\\
\textbf{方法一}\ 利用正弦定理“角化边”:$ a^2\leq b^2+c^2-bc $,所以
\begin{gather*}
    \cos A=\dfrac{b^2+c^2-a^2}{2bc}\geq \dfrac{1}{2},
    \q 0<A\leq \dfrac{\pi}{3}
\end{gather*}
\textbf{方法二}\ 利用$ \sin A=\sin(B+C)=\sin B\cos C+\cos B\sin C $,
\begin{gather*}
    \sin^2B\cos^2C+2\sin B\cos B\sin C\cos C+\sin^2C\cos^2B
    \leq \sin^2B+\sin^2C-\sin B\sin C \\
    \sin B\sin C(2\cos B\cos C+1)\leq 2\sin^2B \sin^2C \\
    \cos(B+C)\leq -\dfrac{1}{2} 
\end{gather*}
剩余略。此法计算量较大,不推荐。

\item $ ^* $正实数$ x,y,z $满足$ \left\{ \begin{aligned} x^2+xy+y^2 =&\ 6 \\
    y^2+yz+z^2=&\ 7  \\
    z^2+zx+x^2=&\ 8  \end{aligned} \right. $,
求$ xy+yz+zx $和$ x+y+z $的值。
\ifteach \\ \textbf{解}\ 
$ a^2=6,b^2=7,c^2=8 $,\\
$ u=a^4+b^4+c^4=149,\ v=a^2b^2+b^2c^2+c^2a^2=146 ,\ w=a^2+b^2+c^2=21 $,\\
$ xy+yz+zx=\sqrt{\dfrac{-u+2v}{3}}=\sqrt{\dfrac{143}{3}}=6.904105\cdots $ \\
$ x+y+z=\sqrt{\dfrac{1}{2}[w+\sqrt{3(-u+2v)}]}=
\sqrt{\dfrac{1}{2}(21+\sqrt{429})}=4.566854\cdots $. 
\fi

\item 在$ \Delta ABC $中,已知$ AB=7,AC=8,BC=9 $,那么它的面积为\underline{
    \hspace{1cm}},内切圆半径为\underline{\hspace{1cm}},外接圆半径为\underline{
    \hspace{1cm}},$ \vec{AB}\cdot \vec{AC}
=$\underline{\hspace{1cm}},$ BC $边的中点为$ D $,$ AD $
的长度为\underline{\hspace{1cm}}.$ I $是它的内心,$ O $是它的外心,
则$ \vec{AO}\cdot\vec{BC} = $\underline{\hspace{2cm}},
$ \vec{AI}\cdot\vec{BC}= $\underline{\hspace{2cm}}.
\ifteach \\
\textbf{解}\ 面积$ S=12\sqrt{5} $,内切圆半径$ r=\sqrt{5}$,外接圆半径$ R=\dfrac{21
    \sqrt{5}}{10} $,$ \vec{AB}\cdot \vec{AC} = 16,
AD=\dfrac{\sqrt{145}}{2} $. 
$ \cos A=\dfrac{2}{7},\ \cos\dfrac{A}{2}=\dfrac{3\sqrt{14}}{14},\ \tan\dfrac{A}{2}=
\dfrac{\sqrt{5}}{3} $,
$ \vec{AO}\cdot\vec{BC}=\dfrac{15}{2} $,
$ \vec{AI}\cdot\vec{BC}=\dfrac{r}{\tan \frac{A}{2}}
\left(|\vec{AC}|-|\vec{AB}|\right)=3 $. 
\fi

\item 已知正方形$ ABCD $的边长为1.当每个$ \lambda_i(i=1,2,3,4,5,6) $取遍
$ \pm 1 $时,求\\ $|\lambda_1\vec{AB}+\lambda_2\vec{BC}+\lambda_3
\vec{CD}+\lambda_4\vec{DA}+\lambda_5 \vec{AC}+\lambda_6\vec{BD}| 
$的最小值是和最大值.
\ifteach \\ \textbf{解}\ 
两条邻边加一条对角线可以合成零向量,或者合成两倍的对角线,最终可以是两倍对角线加
两倍对角线($ 2\sqrt{2}\times \sqrt{2} =4 $),或者是一倍对角线加三倍对角线\\
$ \sqrt{(\sqrt{2})^2+(3\sqrt{2})^2}=2\sqrt{5} $. 
\fi

\item 下图中,$ AB=4,BC \perp AB $,以$ AB $为直径作一个圆,$ BD $是$ \angle ABC $的平分线,$ P $是$ BD $上任意一点,$ EF $是圆$ O $的任意一条直径,求$ \vec{PE}\cdot \vec{PF} $的最小值。
\begin{figure}[h]
    \centering
    \includegraphics[width=0.25\linewidth]{PAxPB最值问题}
\end{figure}
\ifteach \\ \textbf{解}\ 
$ \vec{PE}\cdot \vec{PF}=
PO^2-OF^2 \geq 2-2^2=-2 $. 
\fi

\item 在$ \Delta ABC $中,已知$ \vec{AE}=\dfrac{4}{5}
\vec{AB}+\dfrac{1}{5}\vec{AC},\ \vec{BD}=
\dfrac{1}{3}\vec{BC}+\dfrac{2}{3}\vec{BA} $,
$ AE $与$ BD $交于点$ F $,设$ \vec{BF}=\lambda 
\vec{BD} $,求$ \lambda $. 
\begin{figure}[h]
    \centering
    \includegraphics[width=0.3\linewidth]{长度占比习题2}
\end{figure}
\ifteach \\ \textbf{解}\ 
\begin{figure}[h]
    \centering
    \includegraphics[width=0.3\linewidth]{长度占比习题2-解答}
\end{figure}\\
$ \dfrac{3}{7} $. 过$ D $作$ AE $的平行线交$ BC $于$ J $点,设$ EJ=x $,则$ EC=3x,\ BE=\dfrac{1}{4}EC=\dfrac{3x}{4},\ BJ=\dfrac{7x}{4},\ \dfrac{BF}{BD}=\dfrac{BE}{EJ}=\dfrac{3}{7} $. 
\fi

\item 设$ H $为$ \Delta ABC $的垂心,且有$ \vec{AH}=
\dfrac{1}{2}\vec{AB}+\dfrac{3}{10}\vec{AC} $,
求$ \cos \angle BAC $. 
\ifteach \\ \textbf{解}\ 条件可变形成
\begin{align}\label{垂心条件-变形-习题}
    2\vec{HA}+
    3\vec{HB} +5\vec{HC}=\vec{0}
\end{align}
设 $ \vec{HA}\cdot \vec{HB}=\vec{HA}
\cdot\vec{HC}=\vec{HB}\cdot \vec{HC}=t<0 $,分别用
$ \vec{HB},\vec{HC} $与(\ref{垂心条件-变形-习题})式做数量积,
可得:$ |\vec{HB}|=\sqrt{-\dfrac{7}{3}t} $, $ |\vec{HC}|=\sqrt{-t} $,
$ \cos\angle BHC=\dfrac{\vec{HB}\cdot \vec{HC}}
{|\vec{HB}||\vec{HC}|}=\dfrac{t}{\sqrt{-\frac{7}{3}t}\sqrt{-t}}
=-\dfrac{\sqrt{21}}{7} $,所以$ \cos \angle BAC=\dfrac{\sqrt{21}}{7} $.
\fi

\item 在$ \Delta ABC $中,$ D $是$ BC $边上一点,且$ \vec{BD}=
\dfrac{1}{5}\vec{BC} $. $ E $是$ AD $上一点,$ \vec{AE} =
\dfrac{2}{3}\vec{AD} $. 过$ E $点的直线与$ AB,AC $所在的直线分别交于
$ M,N $两点,设$ \vec{AM}=x\vec{AB},\vec{AN}=
y\vec{AC} $,且$ x>0,y>0 $. \\
(1)求$ \dfrac{8}{x}+\dfrac{2}{y} $的值;(2)求$ xy $的最小值。
\begin{figure}[!htbp]
    \centering
    \includegraphics[width=0.3\linewidth]{系数倒数相加为定值}
\end{figure} 
\ifteach \\ \textbf{解}:
$ \vec{BD}=\mu_1\vec{BC}, \vec{AE} =
\mu_2\vec{AD} $,
\begin{align*}
    \vec{AE} =\mu_2\vec{AD} =&\ \mu_2\left[
    (1-\mu_1)\vec{AB}+\mu_1\vec{AC}\right]\\
    =&\ (1-\mu_1)\mu_2\vec{AB}+\mu_1\mu_2\vec{AC} \\
    \vec{AE}=\lambda\vec{AM}+(1-\lambda)
    \vec{AN}=&\ \lambda x\vec{AB}+(1-\lambda)y
    \vec{AC}
\end{align*}
$ \left\{ 
\begin{aligned}
    &\lambda x = (1-\mu_1)\mu_2 \\
    &(1-\lambda)y = \mu_1\mu_2
\end{aligned} \right. $. 所以$ \dfrac{(1-\mu_1)\mu_2}{x}+\dfrac{\mu_1\mu_2}{y}
=\lambda+1-\lambda=1 $.  \quad $ \dfrac{8}{x}+\dfrac{2}{y}=15 $. \\
(2) $ xy=\dfrac{1}{225}\cdot xy \left(\dfrac{8}{x}+\dfrac{2}{y}\right)^2=
\dfrac{1}{225}\left(\dfrac{64y}{x}+32+\dfrac{4x}{y}\right)\geq\dfrac{64}{225} $. 
\fi

%
%\item 
%\ifteach \\ \textbf{解}\ 
%
%\fi
%
%\item 
%\ifteach \\ \textbf{解}\ 
%
%\fi
%
%\item 
%\ifteach \\ \textbf{解}\ 
%
%\fi
%
%\item 
%\ifteach \\ \textbf{解}\ 
%
%\fi
%
%\item 
%\ifteach \\ \textbf{解}\ 
%
%\fi
%
%\item 
%\ifteach \\ \textbf{解}\ 
%
%\fi

\end{enumerate}
\myfootnote{\CopyrightStatementChap}
% {\footnotesize (可在以下空白区域自行增补知识点。)}  
\cleardoublepage

%~\newpage
%~\newpage

%------------------------------------------


